
\documentclass[11pt,a4paper]{article}
\usepackage[utf8]{inputenc}
\usepackage[T1]{fontenc}
\usepackage[french]{babel}
\usepackage[top=3cm, bottom=2cm, left=2cm, right=2cm]{geometry}
\usepackage{stmaryrd}
\usepackage{amsmath}
\usepackage{amsfonts}
\usepackage{amssymb}
\usepackage{mathrsfs}
\usepackage{amsthm}
\usepackage{layout}
\usepackage{fancyhdr}

\newtheorem*{thm}{Théorème}
\newtheorem{ex}{Exercice}
\newtheorem*{nota}{Notation}
\newtheorem*{rem}{Remarque}
\newtheorem*{rem2}{Remarques}
\newtheorem{de2}{Définition}
\newtheorem{pro2}[de2]{Propriété}
\newtheorem{thm2}[de2]{Théorème}

\setlength{\parindent}{0cm}
\setlength{\parskip}{1ex plus 0.5ex minus 0.2ex}
\newcommand{\hsp}{\hspace{20pt}}
\newcommand{\HRule}{\rule{\linewidth}{0.5mm}}




\title{}

\date{}
\begin{document}


\pagestyle{fancy}

\fancyhead{}
 \fancyfoot{}

 \lhead{ 2023/2024 \\  L3 Mathématiques
}
\chead{\textbf{Calcul formel\\ }} 
 \rhead{  Université de Lorraine \\ }

\newcommand{\lb}{\llbracket}
\newcommand{\rb}{\rrbracket}


\thispagestyle{fancy}

\begin{center}
%    \HRule \\[0.6cm]
    { \huge \bfseries
    Feuille de TD n$^{\boldsymbol{\circ}}$1
     \\ [0cm] }
    \HRule \\[0.5cm]
\end{center}
\

\begin{ex}\
Que renvoie l'algorithme suivant ?

\begin{center}
\begin{tabular}{| l}
Val$(x,y)$\\
si $x>y$ \\
\ \ \ \rm{|} $d \leftarrow x-y$ \\
sinon \\
\ \ \ \rm{|}      $d \leftarrow y-x$\\
renvoyer $d$      \\
\end{tabular}
\end{center}
  
  
\end{ex}

\begin{ex}\
Écrire un algorithme qui, prenant en entrée trois entiers $a,b$ et $c$, détermine le minimum de ces trois valeurs.
\end{ex}


\begin{ex}\
Écrire un algorithme qui, prenant en entrée un réel $a$ et un entier positif $n$, calcule le produit de ces deux valeurs en n'utilisant que l'addition.
\end{ex}




\begin{ex}\
Soit $n$ un entier strictement positif.
\begin{itemize} 
\item[•]On appelle \textbf{diviseur propre} de $n$ un diviseur de $n$ distinct de $n$. 
\item[•] $n$ est dit \textbf{parfait} s'il vaut la somme de ses diviseurs propres.
\item[•]  $n$ est dit \textbf{chanceux} si $(n+m+m^2)$ est premier pour tout $ 0 \leqslant m < n-1$. 
\end{itemize}

\begin{itemize}
\item[$1.$] Écrire un algorithme {Div$($n$)$} qui, prenant en entrée un entier strictement positif $n$, calcule la somme de ses diviseurs propres.
\item[$2.$] Écrire un algorithme Parfait$($n$)$ qui renvoie vrai si l'entier strictement positif $n$ pris en entrée est parfait et faux sinon.
\item[$3.$] Idem pour l'algorithme Chanceux$($n$)$ en supposant qu'on dispose d'un test de primalité Premier$(m)$ qui renvoie vrai si $m$ est premier et faux sinon.   
\end{itemize}

\end{ex}






\end{document}
