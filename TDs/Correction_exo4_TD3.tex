
\documentclass[11pt,a4paper]{article}
\usepackage[utf8]{inputenc}
\usepackage[T1]{fontenc}
\usepackage[french]{babel}
\usepackage[top=3cm, bottom=2cm, left=2cm, right=2cm]{geometry}
\usepackage{stmaryrd}
\usepackage{amsmath}
\usepackage{amsfonts}
\usepackage{amssymb}
\usepackage{mathrsfs}
\usepackage{amsthm}
\usepackage{layout}
\usepackage{fancyhdr}
\usepackage{comment}

\newtheorem*{thm}{Théorème}
\newtheorem{ex}{Exercice}
\newtheorem*{nota}{Notation}
\newtheorem*{remarque}{Remarque}
\newtheorem*{remarques}{Remarques}
\newtheorem*{rem}{Remarque}
\newtheorem*{rem2}{Remarques}
\newtheorem{de2}{Définition}
\newtheorem{pro2}[de2]{Propriété}
\newtheorem{thm2}[de2]{Théorème}

\setlength{\parindent}{0cm}
\setlength{\parskip}{1ex plus 0.5ex minus 0.2ex}
\newcommand{\hsp}{\hspace{20pt}}
\newcommand{\HRule}{\rule{\linewidth}{0.5mm}}





\newcommand{\N}{\mathbb{N}}
\newcommand{\R}{\mathbb{R}}
\newcommand{\Z}{\mathbb{Z}}

\title{}

\date{}
\begin{document}


\pagestyle{fancy}

\fancyhead{}
 \fancyfoot{}

 \lhead{ 2021/2022 \\  L3 Mathématiques
}
\chead{\textbf{ Calcul formel}\\} 
 \rhead{   Université de Lorraine \\ }

\newcommand{\lb}{\llbracket}
\newcommand{\rb}{\rrbracket}


\newcommand{\md}[3]{#1\ \equiv \ #2 \! \! \! \! \! \pmod {#3} }
\newcommand{\nmd}[3]{#1 \not \equiv #2 \! \! \! \! \!  \pmod {#3} }
\newcommand{\mda}[3]{#1 \equiv #2 \! \!  \pmod {#3} }
\newcommand{\nmda}[3]{#1 \not \equiv #2 \! \! \pmod {#3} }
\newcommand{\mo}[2]{#1 \! \! \! \! \! \pmod #2 }
\newcommand{\moa}[2]{#1 \! \!  \pmod {#2} }


\thispagestyle{fancy}


\
\

\textbf{Exercice}

\textit{Pierre donne à sa banque un chèque de $x$ euros et $y$ centimes. Par erreur, le banquier encaisse $y$ euros et $x$ centimes, ce qui représente $5$ centimes de plus que le double du montant de son chèque. Calculer $x$ et $y$.}


\

\textbf{Correction}



On a $(100 y +x) =2(100x+y)+5$, avec $x\in \N$, $y\in \llbracket 0,99\rrbracket$, ce qui équivaut à $199x -98 y=-5$. 

\begin{itemize}
\item[•] Commençons par résoudre cette équation dans $\Z^2$. Soit  $(E_\Z)$ : $199x -98 y=-5$, d'inconnue $(x,y)\in \Z^2$. Effectuons l'algorithme d'Euclide étendu. On a \[\begin{aligned} 199 &=98\times 2+3\\
98&= 3\times 32+2\\ 
3&=2+1.\end{aligned}\] 


On a donc \[\begin{aligned} 1 &= 3-2\\
1 &= 3-(98-32\times 3)\times 3=3\times 33-98\\
 1 &= 33\times (199-2\times 98) -98=33\times 199-67\times 98 \end{aligned}\] En multipliant la dernière égalité par $-5$, on obtient   \[(33\times -5)\times 199-67\times (-5)\times 98=-5=(-165)\times 199-(-365)\times 98=-5,\] Donc $(x_0,y_0)=(-165,-335)$ est une solution particulière de $(E_\Z)$. 

\item[•] Soit $E_0$ l'équation $199x'-98 y'=0$, d'inconnue $(x',y')\in \Z^2$. Résolvons $E_0$. Soit $(x',y')\in \Z^2$. Supposons que $(x',y')$ est solution de $(E_0)$. Alors $199x'=98 y'$ donc $199$ divise $98y'$. Comme $199\wedge 98=1$, on en déduit que $199$ divise $y'$ par le lemme de Gauss. On a donc $y'=199k$, pour un certain $k\in \Z$. On a donc $199 x'=98 y'=98\times 199 k$ donc $x'=98 k$. Donc $(x',y')=(98k,199k)$, pour un certain $k\in \Z$. Réciproquement, si $k\in \Z$, $(98k,199k)$ est solution de $(E_0)$. Donc l'ensemble des solutions de $(E_0)$ est $\{(98k,199k)|k\in \Z\}$.  

\item[•] Soit $(x,y)\in \Z^2$. Alors $(x,y)$ est solution de $(E_\Z)$ si et seulement si $199x-98y=199x_0-98y_0$ si et seulement $199(x-x_0)-98(y-y_0)=0$ si et seulement si il existe $k\in\Z$ tel que ($x-x_0,y-y_0)=(98k,199k)$ si et seulement si $(x,y)=(-165+98k,-335+199k)$, pour un certain $k\in \Z$. 

\item[•] On cherche maintenant les solutions $(x,y)$ de $(E_\Z)$ telles que $x\in \N$ et $y\in\llbracket 0,99 \rrbracket$. On doit alors avoir $k=2$, donc $(x,y)=(31,63)$. 
\end{itemize}


\end{document}
