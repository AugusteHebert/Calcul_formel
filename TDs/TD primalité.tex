%\documentclass[a4paper,12pt]{article}

\documentclass[12pt]{article}
\usepackage[francais]{babel}
\usepackage{cmlgc}
\usepackage{ucs}
\usepackage[utf8]{inputenc}
\usepackage[T1]{fontenc}
%
%\usepackage[english,francais]{babel}
%\usepackage[latin1]{inputenc}
%%\usepackage[utf8]{inputenc}
%\usepackage[T1]{fontenc}

\usepackage{amsmath}
\usepackage{amsfonts}
\usepackage{amssymb}
\usepackage{stmaryrd}
\usepackage{comment}

\newcommand{\df}[1]{{\emph{#1}}}

\newcommand{\C}{\mathbb{C}}
\newcommand{\N}{\mathbb{N}}
\newcommand{\R}{\mathbb{R}}
\newcommand{\Z}{\mathbb{Z}}
\newcommand{\Q}{\mathbb{Q}}


\newcommand{\ja}[2]{\left(\frac{#1}{#2}\right)}
\newcommand{\Hom}{\operatorname{Hom}}
\newcommand{\Aut}{\operatorname{Aut}}
\newcommand{\End}{\operatorname{End}}
\newcommand{\Mat}{\operatorname{Mat}}
\newcommand{\id}{\operatorname{Id}}
\newcommand{\Ne}{\mathbb{N}^*}

\newcommand{\im}{\operatorname{im}}
\newcommand{\tr}{\operatorname{tr}}

\newcommand{\s}{\mathfrak{S}}

\newcommand{\isom}{\stackrel{\sim}{\longrightarrow}}

\newcommand{\inj}{\hookrightarrow}

\newcommand{\SL}{\text{SL}}
\newcommand{\GL}{\text{GL}}
\newcommand{\M}{\text{M}}
\newcommand{\mat}[4]{\begin{pmatrix}
#1 & #2 \\
#3 & #4
\end{pmatrix}}
\newcommand{\eps}{\varepsilon}

%\newtheorem{enonce}{Exercice}
%\newenvironment{exo}[0]{\begin{enonce}\rm}{\smallskip\end{enonce}}
\newcounter{question}
\newcounter{sousquestion}
\newtheorem{enonce}{Exercice}
\newenvironment{exo}[0]{\begin{enonce}{\bf ---}\rm\setcounter{question}{1}}{\end{enonce}}
\newcommand{\quest}{{\setcounter{sousquestion}{1}\vspace{0.1cm}\bf \arabic{question}.\hspace{0.1cm}}\addtocounter{question}{1}}
\newcommand{\sousq}{{\hspace{0.5cm}\vspace{0.1cm}\bf \alph{sousquestion}.\hspace{0.1cm}}\addtocounter{sousquestion}{1}}




\begin{document}

%\noindent \'ENS de Lyon\hfill Option C\\
%2019-2020\hfill \
\smallskip
\begin{center}
{\sc
Tests de primalité}
\end{center}
\smallskip





\begin{exo}{Nombres de Carmichael}

%\quest  Montrer que tout nombre de Carmichael est impair.



\quest  Soit $n\in \N$. On veut montrer que les conditions suivantes sont équivalentes : \begin{enumerate}
\item[(i)] $n$ est sans facteur multiple et $p-1$ divise $n-1$ pour tout facteur premier $p$ de $n$

\item[(ii)] $a^n\equiv a[n]$ pour tout entier $a$

\item[(iii)] $a^{n-1}\equiv 1 [n]$ pour tout entier $a$ premier à $n$.
\end{enumerate}


\sousq Montrer que (i) implique (ii) et que (ii) implique (iii).

\sousq Soit $n$ vérifiant (iii). Supposons que pour un nombre premier $p$, $p^2$ divise $n$. On écrit $n=p^2m$. Montrer que $1+pm$ est d'ordre $p$ modulo $n$. En déduire une contradiction.

\sousq Soit $n$ vérifiant (iii). On écrit $n=p_1\ldots p_r$, où les $p_i$ sont des nombres premiers distincts. Montrer qu'il existe $a\in \llbracket 1,n\rrbracket$ tel que $a[p_i]$ soit une racine primitive $p_i-1$-ième de l'unité pour tout $i$. En déduire que $n$ vérifie (i).


\quest Montrer que tout nombre de Carmichael est produit d'au moins trois nombres premiers distincts.

%\quest Soit $n\in \N$. Montrer que ou bien $n$ ne possède aucun témoin de Fermat, ou bien il en possède au moins $\frac{n}{2}$.
\end{exo}

\begin{comment}
\begin{exo}{Complexité du test de Miller-Rabin}


Soit $n\in \N_{\geq 9}$ un entier impair. On écrit $n-1=2^st$ avec $t$ impair. Soit $a\in \llbracket 2,n-1\rrbracket$. On dit que $a$ est un \textbf{témoin de Miller} (pour $n$) si $a^t\not \equiv 1 [n] $ et $a^{2^i t}\not \equiv -1 [n]$ pour tout $i\in \llbracket 1,s-1\rrbracket$. On rappelle que si $n$ est premier, il n'a pas de témoin de Miller et si $n$ n'est pas premier, au moins les trois-quarts des $a\in \llbracket 2,n-1\rrbracket$ sont des témoins de Miller.

\quest Soit $\epsilon\in ]0,1]$.  Soit $n\in \N$. On tire un nombre $a\in \llbracket 2,n-1\rrbracket $ au hasard et on teste si c'est un témoin de Miller. Majorer le nombre $C$ de $a\in \llbracket 1,n\rrbracket$ qu'il faut tester (avec remise) pour que si $n$ est composé, la « probabilité » de n'avoir choisi aucun témoin de Miller soit inférieure à $\epsilon$.

\quest  Majorer le coût en calcul d'un test de Miller-Rabin prenant en entrée un entier $n$ et renvoyant  « Vrai » si l'entier est composé avec probabilité $< \eps$.
\end{exo}

\begin{exo}{Recherche d'un nombre premier aléatoire}

Soit $m$ un entier. On cherche à obtenir un nombre premier
aléatoire $\leq m$. On suppose qu'on dispose d'un test de primalité
probabiliste, qui répond « Vrai » si l'entier est composé avec probabilité 
$< \eps$. Si $n\in \N$, on note $C(n)$ le temps que met le test à renvoyer une réponse si l'entrée est inférieure ou égale à $n$.

On utilise la méthode suivante : on tire un
nombre $a$ au hasard dans $\{1,\dots,m\}$. Si le test répond Vrai on renvoie 
$a$, sinon on recommence.

\quest Majorer l'espérance du nombre de tirages nécessaires avant d'obtenir un $a$
tel que le test réponde Vrai 
(on se rappellera que  $n$ assez grand, le nombre $\pi(n)$ de
nombres premiers $\leq n$ est de l'ordre de $n/\mathrm{ln} (n)$).

\quest En déduire une majoration de l'espérance du temps que met le test à 
renvoyer un entier $a$.

\quest Majorer la probabilité que l'entier renvoyé soit composé.

\end{exo}
\end{comment}



\begin{exo}{Test de primalité de Lucas-Lehmer pour les nombres de Mersenne}
\end{exo}

\quest Soient $a$ et $n\in \N_{\geq 2}$ tels que $a^n-1$ est premier. Montrer que $a=2$ et que $n$ est premier.

\medskip

Soient $p$ un nombre premier impair et $M=2^p-1$. On pose $s_0=4$ et $s_{k+1}=s_k^2-2$ pour $k\in \N$. L'objectif est de démontrer le théorème suivant : 

$(*)$ $M$ est premier si et seulement si $s_{p-2}\equiv 0[M]$.

\quest Donner une majoration de la complexité d'un test de primalité de $M$ utilisant $(*)$ en fonction  de la taille de $M$.

On pose $\omega=2+\sqrt{3}$ et $\overline{\omega}=2-\sqrt{3}$.


\quest Vérifier que $s_k=\omega^{2^k}+\overline{\omega}^{2^k}$ pour tout $k\in \N$.

\quest Supposons que $s_{p-2}\equiv 0[M]$. Soit $k\in \N$ tel que $s_{p-2}=kM$. Montrer que $\omega^{2^{p-1}}=kM\omega^{2^{p-2}}-1$.

 \quest On suppose que $M$ n'est pas premier et que $s_{p-2}\equiv 0[M]$. Soit $q$ un facteur premier (impair) de $M$. Soit $X$ l'anneau $\Z[\sqrt{3}]/q\Z[\sqrt{3}]$. Si $x=(a,b)\in \mathbb{F}_q^2$, on note $x=a+b\sqrt{3}$ et on note $\mathbb{F}_q\oplus \mathbb{F}_q \sqrt{3}$ au lieu de $\mathbb{F}_q^2$.


\sousq Soit $\phi:X\rightarrow \mathbb{F}_q\oplus \mathbb{F}_q \sqrt{3}$ définie par \[\phi(a+b\sqrt{3}+q\Z[\sqrt{3}])=a[q]+b[q]\sqrt{3},\] pour tous $a,b\in \Z$. Montrer que $\phi:(X,+)\rightarrow ( \mathbb{F}_q\oplus \mathbb{F}_q \sqrt{3},+)$ est un isomorphisme de groupes. Quel est le cardinal de $X$ ?



 Soit $\pi: \Z[\sqrt{3}]\twoheadrightarrow X$ la projection canonique.




\sousq Montrer que $\pi(\omega)$ est d'ordre $2^p$ dans $X^\times$.

\sousq En déduire que $q=M$.



\quest On suppose maintenant que $M$ est premier. 

\sousq Montrer que $\ja{3}{M}=-1$ (où $\ja{.}{.}$ désigne le symbole de Legendre).



\sousq Montrer que $\ja{2}{M}=1$ (on pourra montrer directement que $2$ est un carré modulo $M$ en partant de $2^p\equiv 1[M]$).

Soient $X=\Z[\sqrt{3}]/M\Z[\sqrt{3}]$ et $\pi:\Z[\sqrt{3}]\twoheadrightarrow X$ la projection canonique. Soit $\sigma=2\sqrt{3}$.

\sousq Montrer que $\pi\big((6+\sigma)^M\big)=\pi(6-\sigma).$

\sousq Montrer que $\omega=\frac{(6+\sigma)^2}{24}$. Montrer que $\pi(\omega^{(M+1)/2})=\pi(-1)$ (on pourra utiliser le fait que $24^{(M-1)/2}=(2^{(M-1)/2})^3.3^{(M-1)/2}$).

\sousq Conclure en multipliant chaque membre de l'égalité par $\overline{\omega}^{(M+1)/4}$.



\end{document}
