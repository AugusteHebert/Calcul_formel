
\documentclass[11pt,a4paper]{article}
\usepackage[utf8]{inputenc}
\usepackage[T1]{fontenc}
\usepackage[french]{babel}
\usepackage[top=3cm, bottom=2cm, left=2cm, right=2cm]{geometry}
\usepackage{stmaryrd}
\usepackage{amsmath}
\usepackage{amsfonts}
\usepackage{amssymb}
\usepackage{mathrsfs}
\usepackage{amsthm}
\usepackage{layout}
\usepackage{fancyhdr}
\usepackage{comment}

\newtheorem*{thm}{Théorème}
\newtheorem{ex}{Exercice}
\newtheorem*{nota}{Notation}
\newtheorem*{rem}{Remarque}
\newtheorem*{rem2}{Remarques}
\newtheorem{de2}{Définition}
\newtheorem{pro2}[de2]{Propriété}
\newtheorem{thm2}[de2]{Théorème}

\setlength{\parindent}{0cm}
\setlength{\parskip}{1ex plus 0.5ex minus 0.2ex}
\newcommand{\hsp}{\hspace{20pt}}
\newcommand{\HRule}{\rule{\linewidth}{0.5mm}}

\title{}

\date{}
\begin{document}


\pagestyle{fancy}

\fancyhead{}
 \fancyfoot{}

 \lhead{ 2020/2021 \\  L3 Mathématiques
}
\chead{\textbf{ Calcul formel}\\} 
 \rhead{  Université de Lorraine \\}

\newcommand{\lb}{\llbracket}
\newcommand{\rb}{\rrbracket}

%\newcommand{\md}[3]{#1 \equiv #2 \! \! \! \! \! \pmod {#3} }
\newcommand{\nmd}[3]{#1 \not \equiv #2 \! \! \! \! \!  \pmod #3 }
\newcommand{\mda}[3]{#1 \equiv #2 \! \!  \pmod #3 }
\newcommand{\nmda}[3]{#1 \not \equiv #2 \! \! \pmod #3 }
\newcommand{\mo}[2]{#1 \! \! \! \! \! \pmod #2 }
\newcommand{\N}{\mathbb{N}}
\newcommand{\Z}{\mathbb{Z}}
\newcommand{\md}{\mathrm{\ mod\ }}


\thispagestyle{fancy}

\begin{center}
%    \HRule \\[0.6cm]
    { \huge \bfseries
Correction
     \\ [0cm] }
    \HRule \\[0.5cm]
\end{center}

Exercice~8

1) On a $(a^{2^{j}t})^2=a^{2.2^jt}=a^{2^{j+1}t}=\overline{1}$.

2) Soit $a\in Y$. Soit $E(a)=\{j'\in \llbracket 1,s\rrbracket| a^{2^{j'}t}=\overline{1}\}$. Écrivons $ed-1=k\varphi$, avec $k\in\N^*$. Alors par le petit théorème de Fermat, $a^{k\varphi}=\overline{1}$, donc $a^{2^st}=\overline{1}$, donc $s\in E(a)$. Soit $j=\min E(a)$. Alors par définition, $a\in Y_j$. Par la question 1), $E(a)=\llbracket j,s\rrbracket$. On en déduit que si $a\in Y_{j'}$ pour $j'\in \llbracket 1,s\rrbracket$, alors $j'=j$. On a donc $Y=\bigsqcup_{j=1}^s Y_j$. Soit $a\in Z$. Par définition de $Z$, on a $Z=\bigcup_{j=1}^s Z_j$. De plus, si $j\in \llbracket 1,s\rrbracket$, on a $Z_j\subset Y_j$, donc l'union est disjointe : $Z=\bigsqcup_{j=1}^s Z_j$.


3)a) Comme $Z_j\subset Y_j$, on a $Z_j\subsetneq Y_j$. Soit $a\in Y_j\setminus Z_j$, alors $a^{2^{j-1}t}\in \{\overline{-1},\overline{1}\}$ et $a^{2^{j-1}t}\neq \overline{1}$, donc $a^{2^{j-1}t}=\overline{-1}$.

b) Soit $\phi:\Z/n\Z\rightarrow \Z/p\Z\times \Z/q\Z$ l'isomorphisme chinois. Écrivons $\phi(a)=(a_p,a_q)$. Alors $(a_p)^{2^j t}=-1\md p$ et $(a_q)^{2^jt}=-1\md q$. On a $(a_p)^{2^jtt_p}=(-1\md p)^{t_p}=-1\md p$. Par le petit théorème chinois, $(a^{tt_p})^{2^{k_p}}=(a^t)^{t_pk_p}=1\md p$. Par la question 1), pour tout $k'\geq k_p$, on a $(a^t)^{t_pk'}=1\md p$. On a donc $j\leq k_p$. 

c) Soit $g\in (\Z/p\Z)^\times$ un générateur de $(\Z/p\Z)^\times$. Soit $x=g^{t_p2^{k_p-j}}$. Alors $x$ est d'ordre $2^j$. En effet, $x^{2^j}=1\md p$, donc l'ordre de $x$ divise $2^j$. Comme $g$ est d'ordre $t_p2^{k_p}$, $g^{t_p 2^{k_p-1}}\neq 1\md$, donc $x^{2^{j-1}}\neq 1\md p$, donc $x$ n'est pas d'ordre $2^{j-1}$ donc $x$ est d'ordre $2^j$.   Soit $\epsilon_j\in \Z$ tel que $\epsilon_j\md n= \phi^{-1}\big((x,1\md q)\big)$. Alors $\epsilon_j$ satisfait aux conditions de la question.

d) Écrivons $\phi\big((\epsilon_j\md n)^{2^{j-1}})\big)=(y\md p, 1\md q)$, pour $y\in \Z$. Alors $(y\md p, 1\md q)^2=(1\md p,1\md q)$, donc $(y\md p)^2=1\md p$. On en déduit que $y\md p\in \{1\md p,-1\md p\}$. Par choix de $\epsilon_j$, on sait que $y\md p\neq 1\md p$ (sinon on aurait que $\phi\big((\epsilon_j\md n)^{2^{j-1}}\big)=(1\md p,1\md q)$,  ce qui serait contradictoire avec le choix de $\epsilon_j$). 

e) On a $\phi\big((\epsilon_j a)^{2^{j-1}}\big)=\big((\epsilon_j\md p)^{2^{j-1}}(a_p)^{2^{j-1}},(a_q)^{2^{j-1}}\big)=(-(a_p)^{2^{j-1}},(a_q)^{2^{j-1}})=(1\md p,-1\md q)$. On a donc $(\epsilon_j a)^{2^{j-1}}\notin \{-1\md n,1\md n\}$. On a aussi $\phi\big((\epsilon_j a)^{2^{j}}\big)=\big((\epsilon_j\md p)^{2^{j}}(a_p)^{2^{j}},(a_q)^{2^{j}}\big)=\big((a_p)^{2^{j}},(a_q)^{2^{j}}\big)=(1\md p,1\md q)$, donc $(\epsilon_j a)^{2^{j}}=1\md n$, donc $\epsilon_j a\in Z_j$. 

f) Soit $a\in Y_j\setminus Z_j$, alors $\epsilon_j a \neq a$ et comme $\epsilon_j\md n\in (\Z/n\Z)^\times$, on a $\epsilon_j b\neq \epsilon_j a$ pour tous $b\neq a$. On a donc $\bigsqcup_{a\in Y_j\setminus Z_j} \{\epsilon_j a\}\subset  Z_j$, donc $|Y_j\setminus Z_j|\leq |Z_j|$ donc $|Y_j|\leq 2 |Z_j|$. Si $Y_j=Z_j$, alors $|Y_j|\leq |Z_j|$. On a donc $|Y|=\sum_{j=1}^s |Y_j|\leq \sum_{j=1}^s 2|Z_j|=2|Z|$.



\end{document}
