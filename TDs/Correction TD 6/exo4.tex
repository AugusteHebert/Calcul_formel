
\documentclass[11pt,a4paper]{article}
\usepackage[utf8]{inputenc}
\usepackage[T1]{fontenc}
\usepackage[french]{babel}
\usepackage[top=3cm, bottom=2cm, left=2cm, right=2cm]{geometry}
\usepackage{stmaryrd}
\usepackage{amsmath}
\usepackage{amsfonts}
\usepackage{amssymb}
\usepackage{mathrsfs}
\usepackage{amsthm}
\usepackage{layout}
\usepackage{fancyhdr}
\usepackage{comment}
\usepackage{graphicx}

\newtheorem*{thm}{Théorème}
\newtheorem{ex}{Exercice}
\newtheorem*{nota}{Notation}
\newtheorem*{rem}{Remarque}
\newtheorem*{rem2}{Remarques}
\newtheorem{de2}{Définition}
\newtheorem{pro2}[de2]{Propriété}
\newtheorem{thm2}[de2]{Théorème}

\setlength{\parindent}{0cm}
\setlength{\parskip}{1ex plus 0.5ex minus 0.2ex}
\newcommand{\hsp}{\hspace{20pt}}
\newcommand{\HRule}{\rule{\linewidth}{0.5mm}}

\title{}

\date{}
\begin{document}


\pagestyle{fancy}

\fancyhead{}
 \fancyfoot{}

 \lhead{ 2020/2021 \\  L3 Mathématiques
}
\chead{\textbf{ Calcul formel}\\} 
 \rhead{  Université de Lorraine \\}

\newcommand{\lb}{\llbracket}
\newcommand{\rb}{\rrbracket}

%\newcommand{\md}[3]{#1 \equiv #2 \! \! \! \! \! \pmod {#3} }
\newcommand{\nmd}[3]{#1 \not \equiv #2 \! \! \! \! \!  \pmod #3 }
\newcommand{\mda}[3]{#1 \equiv #2 \! \!  \pmod #3 }
\newcommand{\nmda}[3]{#1 \not \equiv #2 \! \! \pmod #3 }
\newcommand{\mo}[2]{#1 \! \! \! \! \! \pmod #2 }
\newcommand{\N}{\mathbb{N}}
\newcommand{\Z}{\mathbb{Z}}
\newcommand{\md}{\mathrm{\ mod\ }}


\thispagestyle{fancy}

\begin{center}
%    \HRule \\[0.6cm]
    { \huge \bfseries
    Feuille de TD n$^{\boldsymbol{\circ}}$6
     \\ [0cm] }
    \HRule \\[0.5cm]
\end{center}



Exercice 4


1. On a $\varphi(p^\alpha)=|\{x\in \llbracket 0,p^\alpha-1\rrbracket| x\wedge p^\alpha=1\}|$. Si $x\in \Z$, $x\wedge p^\alpha=1$ si et seulement si $x\wedge p=1$ (car $p$ divise tout diviseur non trivial de $p^\alpha$). On a donc \[\varphi(p^\alpha)=|\llbracket 0,p^\alpha-1\rrbracket\setminus p\llbracket 0,p^{\alpha-1}-1\rrbracket|=p^\alpha-p^{\alpha-1}. \]

2. L'ensemble des diviseurs de $p^\alpha$ est $\{p^\beta|\beta\in \llbracket 0,\alpha\rrbracket\}$. On a donc \[\begin{aligned}\sum_{d|p^\alpha} \varphi(d) &=\varphi(1)+\sum_{\beta=1}^\alpha \varphi(p^\beta)\\ &= 1+\sum_{\beta=1}^\alpha p^\beta-p^{\beta-1}\\ 
&= 1+(p-1)+(p^2-p)+(p^3-p^2)+\ldots (p^{\alpha }-p^{\alpha-1})=1+p^\alpha-1=p^\alpha.\end{aligned}\]


3. Si $n\in \N_{\geq 2}$, on note $k(n)$ le nombre de nombres premiers divisant $n$. On a montré que si $k(n)=1$, alors la formule est vraie. Soit $k\in \N$. Soit $\mathscr{P}_k:$ ``Pour tout $n\in \N$ tel que $k(n)\leq k$, $\sum_{d|n}\varphi(d)=n$''. Soit $k\in \N$ tel que $\mathscr{P}_k$ est vraie. Soit $n\in \N$ tel que $k(n)=k+1$. Écrivons $n=p_1^{\alpha_1}\ldots p_{k+1}^{\alpha_{k+1}}$ la décomposition de $n$ en produit de facteurs premiers (les $p_i$ sont des nombres premiers distincts, et les $\alpha_i$ sont des entiers strictement positifs). Soit $n'=n/p_{k+1}^{\alpha_{k+1}}=p_1^{\alpha_1}\ldots p_{k}^{\alpha_{k}}$. Soit $d$ un diviseur de $n$. Alors $d=p_1^{\beta_1}\ldots p_{k+1}^{\beta_{k+1}}$, où $\beta_i\in \llbracket 0,\alpha_i\rrbracket$ pour tout $i\in \llbracket 1,k+1\rrbracket$. Soit $d'=p_1^{\beta_1}\ldots p_k^{\beta_k}$ et $d_{k+1}=p_{k+1}^{\beta_{k+1}}$. Alors $d=d'd_{k+1}$, $d'|n'$ et $d_{k+1}|p_{k+1}^{\alpha_{k+1}}$. Réciproquement, si $d'$ est un diviseur de $n'$ et $d_{k+1}$ est un diviseur de $p_{k+1}^{\alpha_{k+1}}$, alors $d'd_{k+1}$ divise $n$. Ainsi, \[\begin{aligned} \sum_{d|n} \varphi(d) &=\sum_{d'|n',d_{k+1}|p_{k+1}^{\alpha_{k+1}}} \varphi(d'd_{k+1})\\ &= 
 \sum_{d'|n',d_{k+1}|p_{k+1}^{\alpha_{k+1}}} \varphi(d')\varphi(d_{k+1})\text{ car }d'\wedge d_{k+1}=1\text{ si } d'|n'\text{ et } d_{k+1}|p_{k+1}^{\alpha_{k+1}}\\  
 &= \sum_{d'|n'} \varphi(d')\sum_{d_{k+1}|p_{k+1}^{\alpha_{k+1}}} \varphi(d_{k+1})\\
 &= n' p_{k+1}^{\alpha_{k+1}}=n.\end{aligned}\]
 
On en déduit que $\mathscr{P}_{k+1}$ est vraie, donc $\mathscr{P}_k$ est vraie pour tout $k\in \N$ donc la formule est vraie.


\begin{comment}  
Si $P\in \Z[X]$, $P$ est dit \textbf{irréductible} si pour tous $Q,R\in \Z[X]$ tels que $P=QR$, alors $Q$ ou $R$ est inversible dans $\Z[X]$.



\begin{ex}(critère d'Eisenstein)\label{critere_dEisenstein}
\begin{enumerate}

\item Soit $A$ un anneau intègre et $n\in \N$. Soient $Q,R\in A[X]$ tels $QR=X^n$. Montrer qu'il existe $u\in A^\times$ et $k\in \llbracket 0,n\rrbracket$ tels que $Q=uX^{k}$ et $R=u^{-1} X^{n-k}$.

\item Soit $P=\sum_{i=0}^n a_n X^n\in \Z[X]$ un polynôme primitif (c'est à dire que le PGCD de tous ses coefficients vaut $1$). On suppose qu'il existe $p\in \mathbb{P}$ tel que :\begin{itemize}
\item $p| a_i$, pour tout $i\in\llbracket 0,n-1\rrbracket$,

\item $p\not | a_n$,

\item $p^2|a_0$. 
\end{itemize} 


\end{enumerate}


Montrer que $P$ est irréductible.

\item Montrer que $2X^3+3X^2+3$ est irréductible dans $\Z[X]$.

\textit{Indications : (1) on pourra écrire $Q$ et $R$ sous la forme $Q=X^{k_0} \tilde{Q}$, $R=X^{\ell_0}\tilde{R}$, où $k_0,\ell_0\in \N$, $\tilde{Q},\tilde{R}\in A[X]$ sont tels que $\tilde{Q}(0)\neq 0$, $\tilde{R}(0)\neq 0$, et considérer les degrés de $Q$ et $R$.}

\textit{(2) on pourra considérer se placer dans $\Z/p\Z[X]$.}


Exercice~\ref{critere_dEisenstein}
(1) On a $QR=X^{k_0+\ell_0}\tilde{Q}\tilde{R}$, avec $\tilde{Q}(0)\neq 0$ et $\tilde{R}(0)\neq 0$, donc $\tilde{Q}\tilde{R}(0)\neq 0$. On en déduit que le coefficient de degré $k_0+\ell_0$ de $QR$ est non nul, donc $k_0+\ell_0=n$. Comme $A$ est intègre, $\deg(QR)=\deg(Q)+\deg(R)$. De plus, $\deg(Q)\geq k_0$, $\deg(R)\geq \ell_0$, donc $k_0+\ell_0\leq \deg(Q)+\deg(R)=n=k_0+\ell_0$. On en déduit que $k_0=\deg(Q)$, $\ell_0=\deg(R)$, donc il existe $u,v\in A$ tels que $Q=u X^{k_0}$ et $R=vX^{\ell_0}=vX^{n-k_0}$. On a $QR=uv X^{n}=X^n$ donc $uv=1$, d'où le résultat.


(2) Supposons que $P=QR$, où $Q$ et $R$ ne sont pas inversibles dans $\Z[X]$. Comme $P$ est primitif, si $Q\in \Z$, alors $Q=\pm 1$ (car $Q$ divise tous les coefficients de $P$), donc $Q$ est inversible : c'est absurdre. On en déduit que $Q,R$ sont de degré strictement positifs. Pour $S\in \Z[X]$, on note $\overline{S}$ son image dans $\Z/p\Z[X]$. Alors $\overline{\cdot}:\Z[X]\rightarrow \Z/p\Z[X]$ est un morphisme d'anneaux et $\overline{P}=\overline{Q}\overline{R}$. Par hypothèse, $\overline{P}=\overline{X^n}$. On en déduit que $\overline{Q},\overline{R}$ sont des diviseurs de $\overline{X}^n$, donc il existe $u\in \Z/p\Z^*$ et $k\in \llbracket 1,n-1\rrbracket$ tels que $\overline{Q}=u \overline{X^k}$ et $\overline{R}=u^{-1} \overline{X^{n-k}}$. Alors $\overline{Q}(0)=0=\overline{R}(0)$, donc $p$ divise $Q(0)$ et $R(0)$ donc $p^2$ divise $P(0)=QR(0)=a_0$. On aboutit à une contradiction. Donc $P$ est irréductible.

\end{ex}


\begin{ex}\label{cyclicité_groupe_multiplicatif_corps_fini}(Cyclicité des sous-groupes finis du groupe multiplicatif d'un corps, voir Perrin, \textit{cours d'algèbre}, théorème 2.7 page 74)

Soient $K$ un corps (commutatif) et $G$ un sous-groupe fini de $(K,.)$. Soit $n$ l'ordre de $G$. L'objectif est de montrer que $G$ est isomorphe à $\Z/n\Z$. Ce résultat s'applique en particulier à $K=\Z/p\Z$, où $p$ est un nombre premier.

\begin{enumerate}

\item Soit $x\in G$. Montrer que son ordre $d$ divise $n$.

\item Montrer que $\langle x\rangle=\{y\in K|y^d=1\}$ (on pourra utiliser l'exercice~\ref{nombre_racines_polynôme}).

\item Montrer que l'ensemble des éléments dont l'ordre divise $d$ est inclus dans $\langle x\rangle$.

\item Soit $d\in\N$. Montrer que $\Z/d\Z$ a exactement $\varphi(d)$ éléments d'ordre $d$.

\item Si $d$ est un diviseur de $n$, on note $N_d$ le nombre d'éléments de $G$ d'ordre $d$. Montrer que $N_d$ est soit nul, soit égal à $\varphi(d)$.

\item Montrer que $n=\sum_{d|n} N_d$. 

\item En déduire que $N_d=\varphi(d)$ pour tout $d|n$ (on pourra utiliser l'exercice~\ref{indicatrice_d_Euler}), puis que $G$ est isomorphe à $\Z/n\Z$.

\end{enumerate}

\end{ex}

Exercice~\ref{cyclicité_groupe_multiplicatif_corps_fini}

1. Par le théorème de Lagrange, $d=\langle x\rangle$ divise $n$.

2,3. Soit $P=X^d-1\in K[X]$. Soit $y\in \langle x\rangle$. On a $y=x^k$, pour un certain $k\in \llbracket 0,d-1\rrbracket$. Alors $y^d=x^{kd}=(x^d)^k=1$, donc $\langle x\rangle\subset P^{-1}(\{0\})$. D'après l'exercice~\ref{nombre_racines_polynôme}, $P$ a au plus $d$ racines. Comme $|\langle x\rangle |=d$, on a $P^{-1}(\{0\})=\langle x\rangle$. L'ensemble des éléments dont l'ordre divise $d$ étant exactement l'ensemble des racines de $P$, on a le résultat.

4. Soit $a\in \Z/d\Z$. Supposons que $a$ est d'ordre $d$ (dans $(\Z/d\Z,+)$). Alors $|\langle a\rangle |=|\{ka|k\in \Z\}|=|\Z/d\Z|$, donc $ \langle a\rangle =\Z/d\Z$. En particulier, $\overline{1}\in \langle a\rangle$ donc il existe $k\in \Z$ tel que $ka=\overline{1}$, donc $a$ est inversible dans $\Z/d\Z$ (pour $.$). Réciproquement, si $a\in \Z/d\Z^*$, alors il existe $k\in \Z$ tel que $ka=\overline{1}$. Si $\overline{x}\in \Z/d\Z$, alors $xka=\overline{x}$, donc $\langle a\rangle =\Z/d\Z$. Ainsi, l'ensemble des éléments d'ordre $d$ de $\Z/d\Z$ est l'ensemble des inversibles de $(\Z/d\Z,.)$, d'où le résultat.

5. Soit $d$ un diviseur de $n$ tel que $N_d\neq 0$. Soit $X_d$ l'ensemble des éléments de $G$ d'ordre $d$. Soit $x\in X_d$. On a vu que $X_d\subset \langle x\rangle$.  De plus, $\langle x\rangle \simeq \Z/d\Z$, donc $\langle x \rangle$ a exactement $\varphi(d)$ éléments d'ordre $d$. On en déduit que $|X_d|=\varphi(d)$. 

6. Tout élément de $G$ a un ordre divisant $n$ donc $\bigsqcup_{d|n} X_d=G$ donc \[|\bigsqcup_{d|n} X_d|=\sum_{d|n} |X_d|=\sum_{d|n}N_d=n.\]


7. On a vu que $N_d\leq\varphi(d)$ pour tout $d|n$. De plus, on a vu à l'exercice~\ref{indicatrice_d_Euler} que $\sum_{d|n} \varphi(d)=n$. S'il existait $d'$ divisant $n$ tel que $N_{d'}<\varphi(d')$, on aurait nécessairement $n=\sum_{d|n}N_d <\sum_{d|n}\varphi(d)=n$, donc pour tout $d$ divisant $n$, $N_d=\varphi(d)$. En particulier, $N_n=\varphi(n)>0$. Il existe donc $x\in G$ d'ordre $n$. Alors $\Z/n\Z\simeq \langle x\rangle =G$, donc $G$ est cyclique.
\end{comment}

\end{document}
