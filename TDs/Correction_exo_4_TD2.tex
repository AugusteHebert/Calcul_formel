
\documentclass[11pt,a4paper]{article}
\usepackage[utf8]{inputenc}
\usepackage[T1]{fontenc}
\usepackage[french]{babel}
\usepackage[top=3cm, bottom=2cm, left=2cm, right=2cm]{geometry}
\usepackage{stmaryrd}
\usepackage{amsmath}
\usepackage{amsfonts}
\usepackage{amssymb}
\usepackage{mathrsfs}
\usepackage{amsthm}
\usepackage{layout}
\usepackage{fancyhdr}
\usepackage{comment}

\newtheorem*{thm}{Théorème}
\newtheorem{ex}{Exercice}
\newtheorem*{nota}{Notation}
\newtheorem*{remarque}{Remarque}
\newtheorem*{remarques}{Remarques}
\newtheorem*{rem}{Remarque}
\newtheorem*{rem2}{Remarques}
\newtheorem{de2}{Définition}
\newtheorem{pro2}[de2]{Propriété}
\newtheorem{thm2}[de2]{Théorème}

\setlength{\parindent}{0cm}
\setlength{\parskip}{1ex plus 0.5ex minus 0.2ex}
\newcommand{\hsp}{\hspace{20pt}}
\newcommand{\HRule}{\rule{\linewidth}{0.5mm}}





\newcommand{\N}{\mathbb{N}}
\newcommand{\R}{\mathbb{R}}
\newcommand{\Z}{\mathbb{Z}}

\title{}

\date{}
\begin{document}


\pagestyle{fancy}

\fancyhead{}
 \fancyfoot{}

 \lhead{ 2023/2024 \\  L3 Mathématiques
}
\chead{\textbf{ Calcul formel}\\} 
 \rhead{   Université de Lorraine \\ }

\newcommand{\lb}{\llbracket}
\newcommand{\rb}{\rrbracket}


\newcommand{\md}[3]{#1\ \equiv \ #2 \! \! \! \! \! \pmod {#3} }
\newcommand{\nmd}[3]{#1 \not \equiv #2 \! \! \! \! \!  \pmod {#3} }
\newcommand{\mda}[3]{#1 \equiv #2 \! \!  \pmod {#3} }
\newcommand{\nmda}[3]{#1 \not \equiv #2 \! \! \pmod {#3} }
\newcommand{\mo}[2]{#1 \! \! \! \! \! \pmod #2 }
\newcommand{\moa}[2]{#1 \! \!  \pmod #2 }


\thispagestyle{fancy}

\begin{center}
%    \HRule \\[0.6cm]
    { \huge \bfseries
    Exercice 4, Feuille de TD n$^{\boldsymbol{\circ}}$2
     \\ [0cm] }
    \HRule \\[0.5cm]
\end{center}
\



\begin{center}
\begin{tabular}{l}
Pdb$($b,n$)$\\
$ L \leftarrow [\ ]$ \\
tant que $n>0$ \\
\ \ \ {\rm |} $x \leftarrow n \mathrm{\ mod\ }b$ \\
\ \ \ {\rm |}   $L\leftarrow [x]+L$ \\ 
\ \ \ {\rm |}   $n\leftarrow \frac{n-x}{b}$ \\ 
renvoyer $L$.
\end{tabular}
\end{center}

Alors Pdb$(b,n)$ renvoie la liste $[a_\ell,\ldots,a_0]$ telle que :\begin{enumerate}
\item $n=\sum_{i=0}^\ell a_i b^i$,

\item $a_i\in \llbracket 0,b-1\rrbracket$, pour $i\in \llbracket 0,\ell\rrbracket$,

\item $a_\ell\neq 0$.
\end{enumerate}

Notons $x_i, n_i, L_i$ etc. les valeurs de $x$, $n$ et $L$ après $i$ passages dans la boucle « tant que », pour $i\in \N$ tel que cela a un sens.  \begin{itemize}
\item[•] \textbf{Terminaison de l'algorithme}  Soit $i\in \N$ tel que l'étape $i+1$ de l'algorithme (et donc l'étape $i$) est définie (c'est à dire qu'il y a au moins $i+1$ passages dans la boucle « tant que »). Alors $n_{i+1}=(n_i-x_i)/b\leq n_i/b<n_i$, car $n_i>0$ et $b>1$. La suite $(n_j)$ est donc une suite strictement décroissante d'entiers positifs, elle est donc finie, et l'algorithme termine.

\item[•] \textbf{Validité de l'algorithme} Notons $m$ le nombre total de passages dans la boucle « tant que ».  Pour s'inspirer, on peut étudier les étapes $1$ et $2$.  On a   $x_1=n\mathrm{\ mod\ }b=a_0$, donc $L_0=[a_0]$ et \[n_1=\frac{n-a_0}{b}=\frac{\sum_{i=0}^\ell a_i b^i-a_0}{b}=\sum_{i=1}^n a_i b^{i-1}=a_\ell b^{\ell-1}+\ldots + a_2 b+ a_1.\] On a donc $x_2=a_1$, $L_2=[a_1,a_0]$ et \[n_2=\frac{\sum_{i=1}^\ell a_i b^{i-1}-a_1 b}{b}=\sum_{i=2}^\ell a_i b^{i-2}=a_\ell b^{\ell-2}+a_{\ell-1} b^{\ell-3}+\ldots + a_3 b + a_2.\]

    Soit $i\in \llbracket 1,m\rrbracket\cap \llbracket 1,\ell\rrbracket$. Supposons  que $L_i=[a_{i-1},\ldots,a_0]$ et $n_i=\sum_{j=i}^\ell a_j b^{j-i}=a_\ell b^{\ell-i}+\ldots + a_{i}$. Alors $x_{i+1}=a_{i}$, $L_{i+1}=[a_{i},\ldots,a_0]$ et \[n_{i+1}=(a_\ell b^{\ell-i}+\ldots + a_{i+1}b+a_i-a_i)/b=a_{\ell} b^{\ell-(i+1)}+\ldots +a_{i+1}.\] Par récurrence on en déduit : \[\forall j\in \llbracket 1,\ell+1\rrbracket\cap \llbracket 1,m\rrbracket= \llbracket 1,\min(\ell+1,m)\rrbracket, L_j=[a_{j-1},\ldots,a_0], n_j=\sum_{j'=j}^{\ell} a_{j'} b^{j'-j}.\]
    
    
    Comme $n_m=0$ et que $\sum_{j'=j}^\ell a_{j'} b^{j-j'}>0$ pour tout $j\in \llbracket 0,\ell\rrbracket$, on en déduit que $m\geq \ell+1$. Mais comme $n_{\ell+1}=0$, on a $m=\ell+1$ et alors $L_{\ell+1}=L_m=[a_{\ell},\ldots,a_0]$.
    
   
\end{itemize}

(3) Après calcul, on obtient que $146556=\overline{70990}^{12}$.

(4) $\overline{16356}^7=6.1+5.7+3.7^2+6.7^3+1.7^4=4647$.

\end{document}
