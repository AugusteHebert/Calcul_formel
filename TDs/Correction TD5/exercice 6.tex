
\documentclass[11pt,a4paper]{article}
\usepackage[utf8]{inputenc}
\usepackage[T1]{fontenc}
\usepackage[french]{babel}
\usepackage[top=3cm, bottom=2cm, left=2cm, right=2cm]{geometry}
\usepackage{stmaryrd}
\usepackage{amsmath}
\usepackage{amsfonts}
\usepackage{amssymb}
\usepackage{mathrsfs}
\usepackage{amsthm}
\usepackage{layout}
\usepackage{fancyhdr}
\usepackage{comment}
\usepackage{graphicx}

\newtheorem*{thm}{Théorème}
\newtheorem{ex}{Exercice}
\newtheorem*{nota}{Notation}
\newtheorem*{rem}{Remarque}
\newtheorem*{rem2}{Remarques}
\newtheorem{de2}{Définition}
\newtheorem{pro2}[de2]{Propriété}
\newtheorem{thm2}[de2]{Théorème}

\setlength{\parindent}{0cm}
\setlength{\parskip}{1ex plus 0.5ex minus 0.2ex}
\newcommand{\hsp}{\hspace{20pt}}
\newcommand{\HRule}{\rule{\linewidth}{0.5mm}}

\title{}

\date{}
\begin{document}


\pagestyle{fancy}

\fancyhead{}
 \fancyfoot{}

 \lhead{ 2020/2021 \\  L3 Mathématiques
}
\chead{\textbf{ Calcul formel}\\} 
 \rhead{  Université de Lorraine \\}

\newcommand{\lb}{\llbracket}
\newcommand{\rb}{\rrbracket}

%\newcommand{\md}[3]{#1 \equiv #2 \! \! \! \! \! \pmod {#3} }
\newcommand{\nmd}[3]{#1 \not \equiv #2 \! \! \! \! \!  \pmod #3 }
\newcommand{\mda}[3]{#1 \equiv #2 \! \!  \pmod #3 }
\newcommand{\nmda}[3]{#1 \not \equiv #2 \! \! \pmod #3 }
\newcommand{\mo}[2]{#1 \! \! \! \! \! \pmod #2 }
\newcommand{\N}{\mathbb{N}}
\newcommand{\Z}{\mathbb{Z}}
\newcommand{\md}{\mathrm{\ mod\ }}


\thispagestyle{fancy}

\begin{center}
%    \HRule \\[0.6cm]
    { \huge \bfseries
    Feuille de TD n$^{\boldsymbol{\circ}}$5
     \\ [0cm] }
    \HRule \\[0.5cm]
\end{center}

Exercice~6

1. Si $A=\begin{pmatrix}
a &b\\ c& d
\end{pmatrix}\in \mathcal{M}_2(\Z/26\Z)$ est inversible, alors la formule : $A^{-1}=\mathrm{det}(A)^{-1} {^{t\!}}\mathrm{Com}(A)$ s'écrit $A^{-1}=\mathrm{det}(A)^{-1}\begin{pmatrix}
d & -b\\ -c & a
\end{pmatrix}$.


2. $\overline{3}.\overline{9}=\overline{1}$ donc $\overline{3}^{-1}=\overline{9}$. On a donc $A^{-1}=\overline{9}\begin{pmatrix}
 \overline{3} & \overline{-12}\\ \overline{-1} & \overline{5}
\end{pmatrix}=\begin{pmatrix}
\overline{1} &  \overline{22}\\
\overline{17} & \overline{19}
\end{pmatrix}$.

3. On a $\begin{pmatrix}
\overline{2} & \overline{9}\\ \overline{11} & \overline{11}
\end{pmatrix} A=\begin{pmatrix}
\overline{7} & \overline{3}\\ \overline{11} & \overline{23}
\end{pmatrix}$. De plus, $\begin{pmatrix}
\overline{2} & \overline{9}\\ \overline{11} & \overline{11}
\end{pmatrix}$ est inversible, car sont déterminant est $\overline{1}$ qui est inversible. Son inverse est $\begin{pmatrix}
\overline{11} & \overline{-9}\\ \overline{-11} & \overline{2}
\end{pmatrix}$. On a donc $A=\begin{pmatrix}
\overline{11} & \overline{-9}\\
\overline{-11} & \overline{2}
\end{pmatrix}
\begin{pmatrix}
\overline{7} & \overline{3}\\ \overline{11} & \overline{23}
\end{pmatrix}=\begin{pmatrix}
\overline{4} & \overline{8}\\ \overline{23} & \overline{13}
\end{pmatrix}$.

\begin{comment}

\begin{ex}\label{Occurence}\   \\
Écrire un algorithme qui prend en entrée un nombre $n$ et un chiffre  $c$ et  qui calcule le nombre d'occurrences de $c$ dans l'écriture de $n$ en base $10$ (avec le nombre minimal de $0$ possibles) \\
Par exemple, le nombre d'occurrences du chiffre $2$ dans le nombre $321561262$ est $3$.  On supposera que l'on dispose d'un algorithme Ecr\_Dec qui prend en entrée un nombre $n$ et qui renvoie la liste $[a_0,\ldots,a_k]$ telle que les $a_i$ sont dans $\llbracket 0,9\rrbracket$ pour tout $i$, $a_k\neq 0$ et $n=\sum_{i=0}^k a_i 10^i$.
\end{ex}

Exercice~\ref{Occurence}

\begin{tabular}{ll}
\textbf{Algorithme} & Occurence($n,c$)\\
$L \leftarrow $Ecr\_Dec$(n)$\\
$x\leftarrow 0$\\
$\ell\leftarrow \mathrm{len}(L)$\\
pour $i$ de $0$ à $\ell$ :\\
& \ \ \ {\rm | }si $L[i]=c$ : \\
& \ \ \ \ \ \  {\rm | }$x\leftarrow x+1$\\        		
Renvoyer $x$.
\end{tabular}


\begin{ex}\label{Hanoi}$($Tours de Hanoï, voir figure$)$\ \\
Le problème des tours de Hanoï est un jeu de réflexion imaginé par le mathématicien français Édouard Lucas, et consistant à déplacer des disques de diamètres différents d'une tour source à une tour destination en passant par une tour « intermédiaire » et ceci en un minimum de coups, tout en respectant les règles suivantes :
\begin{itemize}
\item on ne peut déplacer plus d'un disque à la fois,
\item on ne peut placer un disque que sur un autre disque plus grand que lui ou sur un emplacement vide.
\end{itemize}
On suppose que cette dernière règle est également respectée dans la configuration de départ.

\begin{itemize}
\item[$1.$]  Essayer de résoudre le problème pour une tour de $1$, $2$, $3$ et $4$ palets.
\item[$2.$]  Décrire une méthode récursive pour résoudre le problèmes des tours de Hanoï.
\end{itemize}

\textit{Indication:}Résoudre le problème des tours de Hanoï, c'est-à-dire déplacer $n$ disques d'une tour source
'S' vers une tour destination 'D' en utilisant une tour intermédiaire 'I', revient à :
\begin{itemize}
\item[-] déplacer $n - 1$ disques de la tour source vers la tour intermédiaire;
\item[-] déplacer $1$ disque de la tour source vers la tour destination;
\item[-] déplacer $n - 1$ disques de la tour intermédiaire vers la tour destination.
\end{itemize}

\end{ex}


Exercice~\ref{Hanoi}

Soit $n\in \N^*$. Supposons que l'on sache déplacer une tour à $n$ disques  d'un bâton à un autre. On considère une tour à $n+1$ disques placés sur $A$. On envoie d'abord les $n$ premiers disques sur $C$. On envoie le plus gros disque sur $B$ puis on déplace les $n$ disques de $C$ sur $B$. On a donc déplacé les $n+1$ disques de $A$ à $B$.


\centering\includegraphics[scale=1]{hanoi.png}
\

\end{comment}


\end{document}
