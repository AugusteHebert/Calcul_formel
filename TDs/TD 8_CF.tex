
\documentclass[11pt,a4paper]{article}
\usepackage[utf8]{inputenc}
\usepackage[T1]{fontenc}
\usepackage[french]{babel}
\usepackage[top=3cm, bottom=2cm, left=2cm, right=2cm]{geometry}
\usepackage{stmaryrd}
\usepackage{amsmath}
\usepackage{amsfonts}
\usepackage{amssymb}
\usepackage{mathrsfs}
\usepackage{amsthm}
\usepackage{layout}
\usepackage{fancyhdr}
\usepackage{comment}
\usepackage{graphicx}

\newtheorem*{thm}{Théorème}
\newtheorem{ex}{Exercice}
\newtheorem*{nota}{Notation}
\newtheorem*{rem}{Remarque}
\newtheorem*{rem2}{Remarques}
\newtheorem{de2}{Définition}
\newtheorem{pro2}[de2]{Propriété}
\newtheorem{thm2}[de2]{Théorème}

\setlength{\parindent}{0cm}
\setlength{\parskip}{1ex plus 0.5ex minus 0.2ex}
\newcommand{\hsp}{\hspace{20pt}}
\newcommand{\HRule}{\rule{\linewidth}{0.5mm}}
\newcommand{\F}{\mathbb{F}}

\title{}

\date{}
\begin{document}


\pagestyle{fancy}

\fancyhead{}
 \fancyfoot{}

 \lhead{ 2025/2026 \\  L3 Mathématiques
}
\chead{\textbf{ Calcul formel}\\} 
 \rhead{  Université de Lorraine \\}

\newcommand{\lb}{\llbracket}
\newcommand{\rb}{\rrbracket}

%\newcommand{\md}[3]{#1 \equiv #2 \! \! \! \! \! \pmod {#3} }
\newcommand{\nmd}[3]{#1 \not \equiv #2 \! \! \! \! \!  \pmod #3 }
\newcommand{\mda}[3]{#1 \equiv #2 \! \!  \pmod #3 }
\newcommand{\nmda}[3]{#1 \not \equiv #2 \! \! \pmod #3 }
\newcommand{\mo}[2]{#1 \! \! \! \! \! \pmod #2 }
\newcommand{\N}{\mathbb{N}}
\newcommand{\Z}{\mathbb{Z}}
\newcommand{\md}{\mathrm{\ mod\ }}


\thispagestyle{fancy}

\begin{center}
%    \HRule \\[0.6cm]
    { \huge \bfseries
    Feuille de TD n$^{\boldsymbol{\circ}}$8
     \\ [0cm] }
    \HRule \\[0.5cm]
\end{center}


\begin{ex}\label{exRacines_carrées}
\begin{enumerate}
\item Déterminer une racine de $\overline{8}$ dans $\Z/23\Z$ et une racine de $\overline{19}$ dans $\Z/31\Z$ (on rappelle que si $p\in \mathbb{P}$ vérifie $p\equiv 3[4]$ et si $\overline{x}\in \Z/p\Z$, alors $(\overline{x}^{(p+1)/4})^2)=\left(\frac{x}{p}\right) \overline{x}$).

\item Déterminer une racine de $\overline{8}$ dans $\Z/23^2\Z$ et une racine de $\overline{19}$ dans $\Z/31^2\Z$ (pour déterminer une racine de $\overline{x}$ dans $\Z/p^2\Z$, avec $p\in \mathbb{P}_{\geq 3}$ alors que l'on connaît une racine  $\overline{\gamma_0}$ de $\overline{x}$ dans $\Z/p\Z$, on peut chercher $\gamma$ sous la forme $\gamma_0+pk$, où $k\in \Z$, calculer $(\gamma_0+pk)^2$ puis essayer d'en déduire une valeur de $k$).

\item Déterminer l'ensemble des  racines de $\overline{33}$ dans $\Z/44\Z$.
\end{enumerate}
\end{ex}

\begin{ex}
\begin{enumerate}
\item Résoudre l'équation $x^2+x-\overline{1}=\overline{0}$, d'inconnue $x\in \Z/31\Z$. 

\item Résoudre l'équation $x^2+x-\overline{1}=\overline{0}$, d'inconnue $x\in \Z/31^2\Z$. 
\end{enumerate} 
\end{ex}



\textbf{Définition : } 
Soient $A$ un anneau et $\omega:A\rightarrow \Z\cup \{\infty\}$. On dit que $\omega$ est une valuation si on a, pour tous $x,x'\in A$:  \begin{enumerate}
\item $\omega(x)=\infty$ si et seulement si $x=0$,

\item $\omega(xx')=\omega(x)+\omega(x')$,

\item $\omega(x+x')\geq \min (\omega(x),\omega(x'))$.
\end{enumerate}



\textbf{Définition : }
Soit $p\in \mathbb{P}$. On définit la valuation $p$-adique $\omega_p:\mathbb{Q}\rightarrow \Z\cup \{\infty\}$ de la façon suivante. Si $x=0$, on pose $\omega_p(x)=-\infty$. Si $x\in \mathbb{Q}\setminus \{0\}$, on écrit $x=p^k x' $, avec $k\in \Z$, $x'=\frac{u}{v}$, avec $(u,v)\in \Z\times \Z\setminus \{0\}$ et $u,v$ premiers avec $p$. On pose alors $\omega_p(x)=k$.

\begin{ex}
\begin{enumerate}
\item Soit $p\in \mathbb{P}$. Montrer que $\omega_p$ est une valuation de $\Z$. 

\item Soit $\omega:K\rightarrow \Z\cup \{\infty\}$, où $K$ est un corps. Soient $x,x'\in K$. On suppose que $\omega(x)\neq \omega(x')$. Que vaut $\omega(x+x')$ ?

\item Soit $K$ un corps. Donner des exemples de valuations non triviales  sur $K[X]$. 

\item Soit $K$ un corps fini. Déterminer l'ensemble des valuations sur $K$.
\end{enumerate}
\end{ex}




\begin{ex}\label{exNombres_racines}
Soit $p\in \mathbb{P}_{\geq 3}$.  Soit $\alpha\in \N^*$.

\begin{enumerate}
\item Soit $a\in \Z$ tel que $a\wedge p=1$. On suppose que $a$ admet une racine modulo $p^\alpha$. Montrer que $\overline{a}$ admet exactement deux racines dans $\Z/p^\alpha\Z$.

\item Déterminer les racines de $\overline{9}$ dans $\Z/27\Z$. Combien-y-en a-t-il ?

\item Soit $a\in \Z\setminus p^\alpha \Z$. On écrit $a=p^k a'$. On suppose que $k$ est pair et que $\left(\frac{a}{p}\right)=1$. Déterminer l'ensemble des racines de $a$ modulo $p^\alpha$, en fonction de celles de $a'$ modulo $p^{\alpha-k}$. Combien $\overline{a}$ admet-il de racines dans $\Z/p^\alpha \Z$ ?
\item Déterminer l'ensemble des racines de $0$ modulo $p^\alpha$. 
\end{enumerate}
\end{ex}

\begin{ex}[Algorithme de Cipolla (1907)]
Soit $p\in \mathbb{P}\setminus\{2\}$. Soit $n\in \F_p^\times$.
\begin{enumerate}
\item Soit $X=\{(a,b)\in \F_p^2\mid a^2-n=b^2\}$. Montrer que $|X|=(p-1)$.

\textit{Indication : on pourra factoriser $a^2-b^2$ et considérer $X_z:=\{(a,b)\in X\mid a+b=z\}$, pour $z\in \F_p$}

\item Soit $\pi:\F_p^2\rightarrow \F_p$ définie par $\pi((x,y)=x$, pour $x\in \F_p$. Montrer que pour tout $a\in \pi(X)$, $|\{b\in \F_p\mid (a,b)\in X\}|=2$.


\item Montrer que $|\{a\in \F\mid a^2-n\text{ n'est pas un carré de }\F_p\}|=(p-1)/2$.


On suppose maintenant que $n$ n'est pas un carré.

\item On choisit $a$ au hasard dans $\F_p^\times $. Quelle est la probabilité que $a^2-n$ ne soit pas un carré de $\F_p$ ? Comment vérifier rapidement que $a^2-n$ n'est pas un carré ?

\item Soit $a\in \F_p^\times$ tel que $a^2-n$ n'est pas un carré de $\F_p$.  Soit $\mathbb{K}=\F_p[X]/(X^2-(a^2-n))$, où $X$ est une indeterminée. Montrer que $\mathbb{K}$ est un corps à $p^2$ éléments.

\item Soit $\sqrt{a^2-n}=\overline{X}$. Que vaut $(\sqrt{a^2-n})^2$ ?

\item Montrer que pour tout $\alpha,\beta\in \mathbb{K}$, on a $(\alpha+\beta)^p=\alpha^p+\beta^p$.

\item Montrer que $(a+\sqrt{a^2-n})^{(p+1)/2}\in \mathbb{K}$ est une racine carrée de $n$. Montrer que c'est en fait un élément de $\F_p$.

\item On pose $p=29$, $n=7$ et $a=3$. On se place dans $\Z/p\Z$. Vérifier que $\overline{7}$ est un carré  et $a^2-n=\overline{2}$ n'en est pas un. Déterminer $(\overline{3}+\overline{\sqrt{2}})^{15}$.
\end{enumerate}

\end{ex}



\begin{ex}
Soit $\mathbb{K}$ un corps, $x\in \mathbb{K}$ et $n\in \N^*$. On dit que $x$ est une racine primitive $n$-ème de l'unité si $x^n=1$ et $x$ est d'ordre $n$ dans $\mathbb{K}^\times$.

 Soit $p$ un nombre premier impair. On veut montrer la deuxième loi complémentaire de la loi de réciprocité quadratique, c'est à dire  que $2$ est un carré dans $\F_p$ si et seulement si $p\equiv \pm 1 [8]$.

\begin{enumerate}





                                                                                                                                   \item Soit $P$ un facteur irréductible de $X^4+1$ dans $\F_p[X]$. Soit $\mathbb{K}=\F_p[X]/(P)$. Montrer que $\mathbb{K}$ est un corps. Quel est son cardinal ?

                                                                                                                                   \item Soit $\zeta=\overline{X}\in \mathbb{K}$. Montrer que $\zeta$ est une racine primitive $8$-ème de l'unité.

                                                                                                                                   \item Montrer que $(\zeta+\zeta^{-1})^2 =2$.

                                                                                                                                   \item Montrer que $\F_p=\{x\in \mathbb{K}\mid x^p=x\}$.


                                                                                                                                   \item Conclure.

                                                                                                                                \end{enumerate}
\end{ex}


\begin{ex}(cf Demazure, \textit{cours d'algèbre}, 5.1.4)
On veut montrer que pour tout $n\in \Z$ impair et  $r\in \N_{\geq 3}$, \begin{equation}\label{e_racines_mod_2r}
(n\text{ est un carré modulo }2^r)\text{si et seulement si } (n\equiv 1[8]).
\end{equation} 

\begin{enumerate}
\item Déterminer l'ensemble des carrés de $\Z/8\Z$. 

\item Soit $n\in \Z$ impair et soit $r\in \N_{\geq 3}$. On suppose que $n$ est un carré modulo $2^r$. Montrer que $n\equiv 1[8]$.

On veut montrer la réciproque, c'est à dire que si $n\in 1+8\Z$, alors $n$ est un carré modulo $2^r$, pour tout $r\in \N$. Soit $n\in 1+8\Z$. On écrit $n=1+8b$, où $b\in \Z$. On cherche une racine $a$ de $n$ sous la forme $a=a(y)=1+4y$, où $y\in \Z$. On définit $N:\Z\rightarrow \Z$ par $N(z)=b-2z^2$, pour $z\in \Z$. 


\item Soit $r\in \N_{\geq 3}$. Montrer que $a(y)^2\equiv n[2^r]$ si et seulement si $N(y)\equiv y[2^{r-3}]$. 

On pose $y_0=0$ et on définit $(y_k)\in \Z^\N$ par $y_{k+1}=N(y_k)$, pour $k\in  \N$. 

\item Soient $k\in \N$ et $\ell\in \N^*$. On suppose que $4^\ell|(N(y_k)-y_k)$.  Montrer que $4^{\ell+1}|(N(y_{k+1})-y_{k+1})$. 

\item Montrer \eqref{e_racines_mod_2r}.

\item Soit $n\in \Z$. On écrit $n=2^k n'$, où $n'\in \Z$ est impair et $k\in \N$. Soit $r\in \N_{\geq k+1}$. On suppose que $r\geq 3$.  Montrer que  $n$ est un carré modulo $2^r$ si et seulement si  les deux conditions suivantes sont satisfaites : \begin{enumerate}
\item $n'\equiv 1 [8]$,

\item $k$ est pair.
\end{enumerate}
\end{enumerate}
\end{ex}



\end{document}
