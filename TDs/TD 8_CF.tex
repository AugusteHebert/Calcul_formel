
\documentclass[11pt,a4paper]{article}
\usepackage[utf8]{inputenc}
\usepackage[T1]{fontenc}
\usepackage[french]{babel}
\usepackage[top=3cm, bottom=2cm, left=2cm, right=2cm]{geometry}
\usepackage{stmaryrd}
\usepackage{amsmath}
\usepackage{amsfonts}
\usepackage{amssymb}
\usepackage{mathrsfs}
\usepackage{amsthm}
\usepackage{layout}
\usepackage{fancyhdr}
\usepackage{comment}
\usepackage{graphicx}

\newtheorem*{thm}{Théorème}
\newtheorem{ex}{Exercice}
\newtheorem*{nota}{Notation}
\newtheorem*{rem}{Remarque}
\newtheorem*{rem2}{Remarques}
\newtheorem{de2}{Définition}
\newtheorem{pro2}[de2]{Propriété}
\newtheorem{thm2}[de2]{Théorème}

\setlength{\parindent}{0cm}
\setlength{\parskip}{1ex plus 0.5ex minus 0.2ex}
\newcommand{\hsp}{\hspace{20pt}}
\newcommand{\HRule}{\rule{\linewidth}{0.5mm}}

\title{}

\date{}
\begin{document}


\pagestyle{fancy}

\fancyhead{}
 \fancyfoot{}

 \lhead{ 2020/2021 \\  L3 Mathématiques
}
\chead{\textbf{ Calcul formel}\\} 
 \rhead{  Université de Lorraine \\}

\newcommand{\lb}{\llbracket}
\newcommand{\rb}{\rrbracket}

%\newcommand{\md}[3]{#1 \equiv #2 \! \! \! \! \! \pmod {#3} }
\newcommand{\nmd}[3]{#1 \not \equiv #2 \! \! \! \! \!  \pmod #3 }
\newcommand{\mda}[3]{#1 \equiv #2 \! \!  \pmod #3 }
\newcommand{\nmda}[3]{#1 \not \equiv #2 \! \! \pmod #3 }
\newcommand{\mo}[2]{#1 \! \! \! \! \! \pmod #2 }
\newcommand{\N}{\mathbb{N}}
\newcommand{\Z}{\mathbb{Z}}
\newcommand{\md}{\mathrm{\ mod\ }}


\thispagestyle{fancy}

\begin{center}
%    \HRule \\[0.6cm]
    { \huge \bfseries
    Feuille de TD n$^{\boldsymbol{\circ}}$6
     \\ [0cm] }
    \HRule \\[0.5cm]
\end{center}

\begin{ex}[Algorithme de Cipolla (1907)]
Soit $p\in \mathbb{P}\setminus\{2\}$. Soit $n\in \F_p$ qui est un carré. Soit $a\in \F_p$ tel que $a^2-n$ n'est pas un carré. Soit $\sqrt{a^2-n}=\overline{X}\in \F_p[X]/(X^2-(a^2-n))=\F_{p^2}$. Montrer que $(a+\sqrt{a^2-n})^{(p+1)/2}\in \F_p$ est une racine carrée de $n$.
\end{ex}

\begin{ex}\label{exRacines_carrées}
\begin{enumerate}
\item Déterminer une racine de $\overline{8}$ dans $\Z/23\Z$ et une racine de $\overline{19}$ dans $\Z/31\Z$ (on rappelle que si $p\in \mathbb{P}$ vérifie $p\equiv 3[4]$ et si $\overline{x}\in \Z/p\Z$, alors $(\overline{x}^{(p+1)/4})^2)=\left(\frac{x}{p}\right) \overline{x}$).

\item Déterminer une racine de $\overline{8}$ dans $\Z/23^2\Z$ et une racine de $\overline{19}$ dans $\Z/31^2\Z$ (pour déterminer une racine de $\overline{x}$ dans $\Z/p^2\Z$, avec $p\in \mathbb{P}_{\geq 3}$ alors que l'on connaît une racine  $\overline{\gamma_0}$ de $\overline{x}$ dans $\Z/p\Z$, on peut chercher $\gamma$ sous la forme $\gamma_0+pk$, où $k\in \Z$, calculer $(\gamma_0+pk)^2$ puis essayer d'en déduire une valeur de $k$).

\item Déterminer une racine de $\overline{33}$ dans $\Z/44\Z$.
\end{enumerate}
\end{ex}

\begin{ex}
\begin{enumerate}
\item Résoudre l'équation $x^2+x-\overline{1}=\overline{0}$, d'inconnue $x\in \Z/31\Z$. 

\item Résoudre l'équation $x^2+x-\overline{1}=\overline{0}$, d'inconnue $x\in \Z/31^2\Z$. 
\end{enumerate} 
\end{ex}

\begin{ex}\label{exNombres_racines}
Soit $p\in \mathbb{P}_{\geq 3}$.  Soit $\alpha\in \N^*$.

\begin{enumerate}
\item Soit $a\in \Z$ tel que $a\wedge p=1$. On suppose que $a$ admet une racine modulo $p^\alpha$. Montrer que $\overline{a}$ admet exactement deux racines dans $\Z/p^\alpha\Z$.

\item Déterminer les racines de $\overline{9}$ dans $\Z/27\Z$. Combien-y-en a-t-il ?

\item Soit $a\in \Z\setminus p^\alpha \Z$. On écrit $a=p^k a'$. On suppose que $k$ est pair et que $\left(\frac{a}{p}\right)=1$. Déterminer l'ensemble des racines de $a$ modulo $p^\alpha$, en fonction de celles de $a'$ modulo $p^{\alpha-k}$. Combien $\overline{a}$ admet-il de racines dans $\Z/p^\alpha \Z$ ?
\item Déterminer l'ensemble des racines de $0$ modulo $p^\alpha$. 
\end{enumerate}
\end{ex}

\end{document}
