\documentclass[11pt,a4paper]{article}
\usepackage[utf8]{inputenc}
\usepackage[T1]{fontenc}
\usepackage[top=2cm, bottom=2cm, left=1.5cm, right=1.5cm]{geometry}
\usepackage{stmaryrd}
\usepackage{graphicx}
\usepackage{amsmath}
\usepackage{amsfonts}
\usepackage{amssymb}
\usepackage{mathrsfs}
\usepackage{amsthm}
\usepackage[french]{babel}
\usepackage{layout}
\usepackage{fancyhdr}
\usepackage{stmaryrd}
\newtheorem*{thm}{Théorème}
\newtheorem{ex}{Exercice}
\newtheorem*{nota}{Notation}
\newtheorem*{rem}{Remarque}
\newtheorem*{rem2}{Remarques}
\newtheorem{de2}{Définition}
\newtheorem{pro2}[de2]{Propriété}
\newtheorem{thm2}[de2]{Théorème}
\setlength{\parindent}{0cm}
\setlength{\parskip}{1ex plus 0.5ex minus 0.2ex}
\newcommand{\hsp}{\hspace{20pt}}
\newcommand{\HRule}{\rule{\linewidth}{0.5mm}}
\title{}
\newcommand{\N}{\mathbb{N}}
\newcommand{\Z}{\mathbb{Z}}


\date{}
\begin{document}

\pagestyle{fancy}

\fancyhead{}
 \fancyfoot{}

 \lhead{ 2023/2024 \\  L3 Mathématiques \\ Semestre 5
}
\chead{\textbf{ Calcul formel}\\} 
 \rhead{\  Université de Lorraine \\}



\newcommand{\lb}{\llbracket}
\newcommand{\rb}{\rrbracket}

\newcommand{\md}[3]{#1 \equiv #2 \! \! \! \! \! \pmod {#3} }
\newcommand{\nmd}[3]{#1 \not \equiv #2 \! \! \! \! \!  \pmod #3 }
\newcommand{\mda}[3]{#1 \equiv #2 \! \!  \pmod #3 }
\newcommand{\nmda}[3]{#1 \not \equiv #2 \! \! \pmod #3 }
\newcommand{\mo}[2]{#1 \! \! \! \! \! \pmod #2 }
\newcommand{\moa}[2]{#1 \! \! \pmod #2 }
\thispagestyle{fancy}

\vspace*{-0.9cm}
\begin{center}
%    \HRule \\[-0cm]
    { \huge \bfseries
    Examen TP
     \\ [0cm] }
    \HRule \\[0.5cm]
\end{center}

\begin{center}
  mercredi 10 janvier 2024 \\
\textbf{Durée: 2 heures}
\end{center}


\subsection*{Instructions}


Penser à numéroter les questions. Vous pouvez ajouter des remarques ou des commentaires à vos algorithmes en créant des parties texte dans le fichier Sage (pour ajouter un commentaire, il faut utiliser \#, qui se situe au niveau du $3$ sur certains claviers).

Les exercices sont indépendants. Ils pourront être traités dans l'ordre que l'on souhaitera. 

On rappelle que si $a,b\in \Z$, $a\wedge b$ s'obtient grâce à la fonction \texttt{gcd}($a,b$).


\begin{ex}(les   questions sont indépendantes)

\begin{enumerate}

\item Écrire un programme \texttt{factorielle}($n$), qui prend en entrée un entier $n$ et qui renvoie $n!$, sans utiliser la fonction \texttt{factorial}.



\item Écrire un programme \texttt{decomposition}($n$) qui prend en entrée un entier naturel $n$ non nul et qui renvoie $(s,t)\in \N$ tels que $n=2^s t$, avec $t$ impair.


\item On admet le théorème de Dirichlet: pour tous $m,n\in \N^*$ tels que $m\wedge n =1$, il existe une infinité de nombres premiers congrus à $m$ modulo $n$. Écrire un programme qui prend en entrée $m,n,K\in \N$, qui renvoie «~erreur~» si $m\wedge n\neq 1$, et qui renvoie la liste des $K$ premiers nombres premiers congrus à $m$ modulo $n$ sinon. Donner la liste des $10$ premiers nombres premiers congrus à $3$ modulo $100$.

\item Un triplet pythagoricien est un élément $(a,b,c)\in \N^3$ tel que $a^2+b^2=c^2$. Déterminer la liste des triplets pythagoriciens $(a,b,c)$ tels que $c\in \llbracket 0,25\rrbracket$.




\item Déterminer (à l'aide de Sage) le reste dans la division euclidienne de $10^{(10^{10})}$ par $7$.


\end{enumerate}
\end{ex}


\begin{ex}
\begin{enumerate}
\item On définit la suite $(u_n)_{n\in \N}$ par $u_0=0$, $u_1=1$ et $u_{n+2}=u_n+u_{n+1}$ pour tout $n\in \N$. Écrire un programme qui prend en entrée un entier $n\in \N$ et qui renvoie $u_n$. 

\item Que vaut le reste de $u_{10^7}$ dans la division euclidienne par $10$ ? (en s'y prenant bien, le temps de calcul de l'ordinateur peut être de l'ordre de quelques secondes).
\end{enumerate}
\end{ex}

\begin{ex}
\begin{enumerate}
\item Écrire un programme qui prend en entrée un entier $n\in \N_{\geq 2}$, et qui renvoie "True" s'il est premier et "False" s'il ne l'est pas, sans utiliser la commande \texttt{is\_prime} (on pourra par exemple tester si les nombres  entre $1$ et $ \sqrt{n}$ divisent $n$).

\item En utilisant la commande "\texttt{is\_prime}"  écrire un programme qui prend en entrée un entier $n$ et qui renvoie le $n$-ième nombre premier. Quel est le $1000$-ème nombre premier ?

\item Écrire un programme qui prend en entrée un élément $n\in \N_{\geq 2}$ et qui renvoie la liste de ses diviseurs premiers sous la forme d'une liste $L=[p_1,\ldots,p_k]$, avec $k\in \N^*$, $p_1< \ldots <p_k$. On pourra utiliser la fonction \texttt{next\_prime} (mais pas la fonction "\texttt{prime\_divisors}").


 \item Écrire un programme qui prend en entrée un élément $n\in \N_{\geq 2}$, et qui renvoie la liste $[(p_1,\alpha_1),\ldots,(p_k,\alpha_k)]$ telle que $k\in \N^*$, $p_1<\ldots<p_k$, les $p_i$ sont premiers, $\alpha_i\in \N^*$ pour tout $i\in \llbracket 1,k\rrbracket$ et $n=p_1^{\alpha_1}\ldots p_k^{\alpha_k}$. On ne  pourra pas utiliser la fonction \texttt{factor} (mais on pourra utiliser la fonction \texttt{prime\_divisors}). 
\end{enumerate}
\end{ex}



\begin{ex}
 On rappelle le théorème suivant.

\begin{thm} 
Soit  $n$ un entier impair $\geq 3$. Alors :\begin{enumerate}
\item $n$ est premier si et seulement si pour tous $a\in \Z$ tel que $a\wedge n=1$, $ \big( \tfrac{a}{n} \big)\equiv a^{(n-1)/2} [n]$,

\item si $n$ est composé, $\{a\in \llbracket 2,n-1\rrbracket|\ a^{(n-1)/2}\equiv \left(\frac{a}{n}\right) [n]\}$ a au plus $\varphi(n)/2$ éléments.
\end{enumerate}
\end{thm}

Si $n$ est composé et si $a\in \llbracket 2,n-1\rrbracket$, on dit que $a$ est un \textbf{témoin d'Euler} si $\left(\frac{a}{n}\right)\not \equiv a^{(n-1)/2}[n]$. Le test de Solovay-Strassen pour un nombre $n$ consiste à choisir $k\in \N$ tel que $2^k$ est «~grand~» et à effectuer les étapes suivantes : \begin{enumerate}
\item le programme choisit un nombre $a$ de $\llbracket 2,n-1\rrbracket$  aléatoirement,

\item il teste si $a$ est un témoin d'Euler,

\item si $a$ est un témoin d'Euler, le programme s'arrête et renvoie "$n$ n'est pas premier", sinon il   recommence les étapes les étapes (a), (b) et (c), sauf s'il a déjà tiré $k$ nombres auquel cas il renvoie "$n$ est peut-être premier".
\end{enumerate}

\begin{enumerate}
\item Écrire un programme qui prend en entrée deux entiers $n,a$ tels que $n\in \N^*$ est impair et $a\in \llbracket 2,n-1\rrbracket$ et qui teste si $a$ est un témoin d'Euler (on pourra utiliser la fonction \texttt{jacobi\_symbol}). 

\item Implémenter le test de Solovay-Strassen, qui prend en entrée un entier $n$ et un entier $k$.

\item Faire la liste des nombres composés impairs $n\in \llbracket 3,10^5\rrbracket$ pour lesquels \texttt{Solovay\_strassen}($n$,2) renvoie $n$ est peut-être premier. Faire la liste des nombres composés impairs $n\in \llbracket 3,10^5\rrbracket$ pour lesquels \texttt{Solovay\_strassen}($n$,5) renvoie $n$ est peut-être premier. 
\end{enumerate}

\end{ex}


\begin{ex}
Le chiffrement de Vigenère fonctionne sur le principe suivant : on dispose d'une clé qui est une chaîne de caractères (en général un mot), et on décale la première lettre du texte d'un rang correspondant à celui de la première lettre de la clé dans l'alphabet et ainsi de suite. Par exemple, si la clé est « python », la première
lettre sera décalée de 15 (p est la 15ème lettre de l'alphabet, en partant de 0), le deuxième de 24 etc. On
revient au début de la clé si le texte à crypter est plus long que la clé).
Exemple :
\begin{tabular}{|c|| c|c|c|c|c|c|c|c|c|c|c|c|c|c|c|c|c|c|}
\hline
texte clair &  u & n&  e & x&  e&  m&  p&  l&  e&  d&  e&  c&  h&  i&  f&  f&  r&  e \\
\hline
clé&  p&  y&  t& h&  o&  n&  p&  y&  t&  h&  o&  n&  p&  y&  t&  h&  o&  n \\
\hline
décalage &  15&  24&  19&  7&  14&  13&  15&  24&  19&  7 & 14 &  13&  15&  24&  19&  7&  14&  13 \\
\hline
texte crypté &  J &  L &  X &  E &  S &  Z  &  E &  J  & X  & K  & S  & P &  W &  G &  Y  & M  & F  & R  \\
\hline
\end{tabular}




\begin{itemize}
\item[$1.$] Écrire un programme prenant comme argument une clé \texttt{cle} et le texte à crypter \texttt{texte} (deux
chaînes de caractères en majuscule et sans espaces) et effectuant le chiffrement.
\item[$2.$] Comment déchiffrer le texte en connaissant la clé ? Écrire le programme.
\end{itemize}

\end{ex}


\begin{ex}
On rappelle le fonctionnement du cryptosystème RSA. Soient $p$ et $q$ deux nombres premiers distincts  et $n=pq$. Soient $\varphi=(p-1)(q-1)$ et $e\in \llbracket 2,\varphi-1\rrbracket$ un élément premier avec $\varphi$. On veut transmettre un message $m\in \llbracket 0,n-1 \rrbracket$. Pour le crypter, on calcule $m^e\mathrm{\ mod\ }n$. Pour déchiffrer un message $C\in \llbracket 0,n-1\rrbracket$, on calcule $C^d\mathrm{\ mod\ }\varphi$, où $d\in \llbracket 2,\varphi-1\rrbracket$ est tel que $ed\equiv 1[\varphi]$.  La clé publique est alors $(n,e)$ et la clé privé est $(d,\varphi)$. 

\begin{enumerate}
\item Écrire un programme \texttt{premier\_aleatoire} qui prend en entrée un entier $K$ et qui renvoie un nombre nombre premier aléatoire entre $10^K$ et $10^{K+1}$. Le résultat devra changer à chaque exécution du programme (sauf malchance). On pourra utiliser la fonction \texttt{ZZ.random\_element}.

\item Écrire un programme \texttt{cle} qui prend en entrée un entier $K$ et qui renvoie une liste $[(n,e)],[d,\varphi)]$, où $n=pq$, $\varphi=(p-1)(q-1)$, $e\in \llbracket 2,\varphi-1\rrbracket$ est tel que $e\wedge \varphi=1$, $d\in \llbracket 2,\varphi-1\rrbracket$ est tel que $ed\equiv 1[\varphi]$ et $p,q$ sont deux nombres premiers aléatoires entre $10^K$ et $10^{K+1}$ (si le programme de la question précédente ne fonctionne pas, on pourra fixer $p=708641$ et $q=817787$).

\item Crypter le message $2\  718\  281\ 828$, avec $p=708641$ et $q=817787$, $n=pq$ et $e=691$.

\item Écrire un programme qui prend en entrée un message $m$, une clé publique $(n,e)$, et qui renvoie le message crypté suivant le cryptosystème RSA.

\item Écrire un programme qui prend en entrée un message $C$ et une clé publique $(n,e)$, et qui renvoie le message décrypté (c'est à dire l'élément $m\in \llbracket 0,n-1\rrbracket$ tel que $m^e\equiv C[n]$. (on pourra utiliser la fonction \texttt{factor} pour déterminer la clé privée à partir de la clé publique :  si $n\in \N$ se décompose en facteur premier $n=p_1^{\alpha_1}\ldots p_k^{\alpha_k}$, \texttt{factor}$(n)[0]=(p_1,\alpha_1)$ et  \texttt{factor}$(n)[0][0]=p_1$).

\item On pose $p=708641$ et $q=817787$ et $n=pq$ et $e=691$. Décrypter le message $279148699747$. 
\end{enumerate}

\end{ex}






\end{document}