
\documentclass[11pt,a4paper]{article}
\usepackage[utf8]{inputenc}
\usepackage[T1]{fontenc}
\usepackage[french]{babel}
\usepackage[top=3cm, bottom=2cm, left=2cm, right=2cm]{geometry}
\usepackage{stmaryrd}
\usepackage{amsmath}
\usepackage{amsfonts}
\usepackage{amssymb}
\usepackage{mathrsfs}
\usepackage{amsthm}
\usepackage{layout}
\usepackage{fancyhdr}

\newtheorem*{thm}{Théorème}
\newtheorem{ex}{Exercice}
\newtheorem*{nota}{Notation}
\newtheorem*{rem}{Remarque}
\newtheorem*{rem2}{Remarques}
\newtheorem{de2}{Définition}
\newtheorem{pro2}[de2]{Propriété}
\newtheorem{thm2}[de2]{Théorème}

\setlength{\parindent}{0cm}
\setlength{\parskip}{1ex plus 0.5ex minus 0.2ex}
\newcommand{\hsp}{\hspace{20pt}}
\newcommand{\HRule}{\rule{\linewidth}{0.5mm}}

\usepackage{comment}
\usepackage{xr-hyper}

\title{}

\date{}



\makeatletter%pour faire référence aux exercices de enonce.tex
\newcommand*{\addFileDependency}[1]{% argument=file name and extension
  \typeout{(#1)}
  \@addtofilelist{#1}
  \IfFileExists{#1}{}{\typeout{No file #1.}}
}
\makeatother

\newcommand*{\myexternaldocument}[1]{%
    \externaldocument{#1}%
    \addFileDependency{#1.tex}%
    \addFileDependency{#1.aux}%
}
\myexternaldocument{DM1}%pour faire référence aux exercices de DM1.tex


\begin{document}




\pagestyle{fancy}

\fancyhead{}
 \fancyfoot{}

 \lhead{ 2022/2023 \\  L3 Mathématiques
}
\chead{\textbf{ Calcul formel}\\} 
 \rhead{ Université de Lorraine \\  }

\newcommand{\lb}{\llbracket}
\newcommand{\rb}{\rrbracket}
\newcommand{\N}{\mathbb{N}}
\newcommand{\Z}{\mathbb{Z}}




\newcommand{\md}[3]{#1\ \equiv \ #2 \! \! \! \! \! \pmod {#3} }
\newcommand{\nmd}[3]{#1 \not \equiv #2 \! \! \! \! \!  \pmod {#3} }
\newcommand{\mda}[3]{#1 \equiv #2 \! \!  \pmod {#3} }
\newcommand{\nmda}[3]{#1 \not \equiv #2 \! \! \pmod {#3} }
\newcommand{\mo}[2]{#1 \! \! \! \! \! \pmod #2 }
\newcommand{\moa}[2]{#1 \! \!  \pmod {#2} }

\thispagestyle{fancy}

\begin{center}
%    \HRule \\[0.6cm]
    { \huge \bfseries
Corrigé DM 1
     \\ [0cm] }
    \HRule \\[0.5cm]
\end{center}

1) Dans $\Z/6\Z$, les idempotents sont $\overline{0}$, $\overline{1}, \overline{3},\overline{4}$. Dans $\Z/7\Z$, les idempotents sont $\overline{0},\overline{1}$.

\eqref{inverse_theorem_chinois} \textbf{Première méthode :} On a $25-6\times 4=1$. On pose donc $c_1=25\mathrm{\ mod\ }100$ et $c_2=-24\mathrm{\ mod\ }100$. On a alors $\phi(c_1)=((25\mathrm{\ mod\ }4,25\mathrm{\ mod\ }25))=(1\mathrm{\ mod\ }4, 0\mathrm{\ mod\ }25)$ et $\phi(c_2)=(-24\mathrm{\ mod\ }4,-24\mathrm{\ mod\ }25)=(0\mathrm{\ mod\ }4,1\mathrm{\ mod\ }25)$. On a donc $\phi^{-1}(1\mathrm{\ mod\ }4, 0\mathrm{\ mod\ } 25)=c_1$, $\phi^{-1}((0\mathrm{\ mod\ }4,1\mathrm{\ mod\ }25))=c_2=76\mathrm{\ mod\ }100$.

\textbf{Deuxième méthode : l'algorithme de Garner.} On cherche $x\in \llbracket 0,99\rrbracket$ tel que $x\equiv 0[4]$ et $x\equiv 1 [25]$. D'après le théorème de la base mixte, on peut écrire $x$ sous la forme $x=\nu_1+25\nu_2$, où $\nu_1\in \llbracket 0,24\rrbracket$ et $\nu_2\in \llbracket 0,3\rrbracket$. Comme $x=\nu_1+25\nu_2\equiv 0[25]$, on a $\nu_1\equiv 0[25]$ et donc $\nu_1=0$. On a donc $x=25\nu_2 \equiv 1[4]\equiv 24\nu_2+\nu_2[2]\equiv \nu_2[4]$, donc $\nu_2=1$. On a donc $x=25$, donc $\phi^{-1}(1\mathrm{\ mod\ }4, 0\mathrm{\ mod\ } 25)=25\mathrm{\ mod\ }100$.

On cherche maintenant $x\in \llbracket 0,99\rrbracket$ tel que $x\equiv 1[4]$ et $x\equiv 0[25]$. On cherche $x$ sous la forme $x=\nu_1+25\nu_2$, où $\nu_1\in \llbracket 0,24\rrbracket$ et $\nu_2\in \llbracket 0,3\rrbracket$. On a $x=\nu_1+25\nu_2\equiv 1[25]$, donc $\nu_1\equiv 1[25]$ donc $\nu_1=1$. On a aussi $x=1+25\nu_2\equiv 0[4]\equiv 1+\nu_2\equiv[4]$ donc $\nu_2\equiv 3[4]$ donc $\nu_2=3$. On a donc $x=76$, donc $\phi^{-1}((0\mathrm{\ mod\ }4,1\mathrm{\ mod\ }25))=76\mathrm{\ mod\ }[100]$.


\textbf{Remarque : }On pourrait aussi chercher $x$ sous la forme $x=\nu_1+4\nu_2$, avec $\nu_1\in\llbracket 0,3\rrbracket$ et $\nu_2\in \llbracket 0,24\rrbracket$. Cependant il est plus judicieux de faire dans l'ordre que l'on a fait car le calcul de $\nu_1$ est immédiat. Pour le calcul de $\nu_2$, il faut inverser un élément de $(\Z/25\Z)^\times$ ($4\mathrm{\ mod\ }25$) ou un élément de $(\Z/4\Z)^\times$ ($25\mathrm{\ mod\ }4$), et il est plus rapide d'inverser un élément de $(\Z/4\Z)^\times$ qu'un élément de $(\Z/25\Z)^\times$. 


\eqref{idempotent_explicite} Soit $x=\phi^{-1}(1\mathrm{\ mod\ }4, 0\mathrm{\ mod\ } 25)$. Alors $\phi(x)^2=(1\mathrm{\ mod\ }4, 0\mathrm{\ mod\ } 25)^2=(1^2\mathrm{\ mod\ }4, 0^2\mathrm{\ mod\ } 25)=(1\mathrm{\ mod\ }4, 0\mathrm{\ mod\ } 25)=\phi(x)$. On a donc $x^2=(\phi^{-1}(\phi(x)))^2=\phi^{-1}(\phi(x)^2)=\phi^{-1}(\phi(x))=x$. Donc $x$ est idempotent. De même, $\phi^{-1}((0\mathrm{\ mod\ }4,1\mathrm{\ mod\ }25))$ est idempotent. 

\textbf{Remarque : }On a en fait montré que l'image d'un idempotent par un isomorphisme est un idempotent.

\eqref{inversibles} On a $(\Z/10\Z)^\times =\{x\mathrm{\ mod\ }10\in \llbracket 0,9\rrbracket | x\wedge 10=1\}=\{1\mathrm{\ mod\ }10,3\mathrm{\ mod\ }10,7\mathrm{\ mod\ }10,9\mathrm{\ mod\ }10\}$ et  \[\begin{aligned}(\Z/16\Z)^\times &=\{x\mathrm{\ mod\ }10\in \llbracket 0,15\rrbracket | x\wedge 15=1\} \\ 
&=\{1\mathrm{\ mod\ }16,3\mathrm{\ mod\ }16, 5\mathrm{\ mod\ }16,7\mathrm{\ mod\ }16,9\mathrm{\ mod\ }16,11\mathrm{\ mod\ }16,13\mathrm{\ mod\ }16,15\mathrm{\ mod\ }16\}.\end{aligned}\]

\eqref{idempotent_anneau} Soit $x\in A^\times$ tel que $x^2=x$. Alors $x^{-1}.x^2=x^{-1}.x=1=x$, donc le seul idempotent inversible de $A$ est $1$.

\eqref{idempotents10_16} À part $1$, on recherche les idempotents parmi les éléments non inversibles. Après calcul, on obtient que l'ensemble des idempotents de $\Z/10\Z$ est $\{0\mathrm{\ mod\ }10, 1\mathrm{\ mod\ }10, 5\mathrm{\ mod\ }10,,6\mathrm{\ mod\ }10\}$ et l'ensemble des idempotents de $\Z/16\Z$ est $\{0\mathrm{\ mod\ }16,1\mathrm{\ mod\ }16\}$. 

\eqref{pgcd} Si $x\in \Z$, on a $x-(x-1)=1$, donc d'après le théorème de Bézout, $x\wedge (x-1)=1$.

\eqref{caracterisation_idempotent_primaire} Soit  $x\mathrm{\ mod\ }p^\alpha \in \Z/p^\alpha\Z$.  On suppose que $(x\mathrm{\ mod\ }p^\alpha)^2=x\mathrm{\ mod\ }p^\alpha$. Alors $\overline{x}(\overline{x}-1)=\overline{0}$ dans $\Z/p^\alpha\Z$.  Alors $x^2-x\mathrm{\ mod\ }p^\alpha=0\mathrm{\ mod\ }p^\alpha=x(x-1)\mathrm{\ mod\ }p^\alpha$. On en déduit que $p^\alpha$ divise $x\wedge (x-1)$. Comme $x\wedge (x-1)=1$, on a $x\wedge p=1$ ou $x-1\wedge p=1$. On a donc $p^\alpha \wedge x=1$ ou $p^\alpha\wedge (x-1)=1$. Dans le premier cas, $\overline{x}$ est inversible donc $\overline{x}-\overline{1}=\overline{0}$, dans le second cas, $\overline{x}-\overline{1}$ est inversible, donc $\overline{x}=\overline{0}$. D'où le résultat.



\eqref{cardinal_idempotents} De même qu'à la question \eqref{idempotent_explicite}, si $x\in \Z/n\Z$, $x$ est idempotent si et seulement si $f(x)$ l'est. Écrivons $f(x)=(x_1,\ldots,x_k)$, avec $x_i\in \Z/p_i^{\alpha_i}\Z$, pour $i\in \llbracket 1,k\rrbracket$. Alors $f(x)^2=(x_1^2,\ldots,x_k^2)$, donc $f(x)^2=f(x)$ si et seulement si $x_i^2=x_i$ pour tout $i\in\llbracket 1,k\rrbracket$. D'après la question \eqref{caracterisation_idempotent_primaire} on en déduit le résultat. Il y a donc $2^k$ idempotents dans $\Z$. 

\eqref{idempotents_100} L'ensemble des idempotents de $\Z/100\Z$ est \[\begin{aligned}&\{\phi^{-1}(0\mathrm{\ mod\ }4,0\mathrm{\ mod\ }25),\phi^{-1}(1\mathrm{\ mod\ }4,0\mathrm{\ mod\ }25),\phi^{-1}(0\mathrm{\ mod\ }4,1\mathrm{\ mod\ }25),\phi^{-1}(1\mathrm{\ mod\ }4,1\mathrm{\ mod\ }25)\} \\&=\{0\mathrm{\ mod\ }100,1\mathrm{\ mod\ }100 ,25\mathrm{\ mod\ }100,76\mathrm{\ mod\ }100\}.\end{aligned}\]



\end{document}