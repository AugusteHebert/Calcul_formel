
\documentclass[11pt,a4paper]{article}
\usepackage[utf8]{inputenc}
\usepackage[T1]{fontenc}
\usepackage[french]{babel}
\usepackage[top=3cm, bottom=2cm, left=2cm, right=2cm]{geometry}
\usepackage{stmaryrd}
\usepackage{amsmath}
\usepackage{amsfonts}
\usepackage{amssymb}
\usepackage{mathrsfs}
\usepackage{amsthm}
\usepackage{layout}
\usepackage{fancyhdr}

\newtheorem*{thm}{Théorème}
\newtheorem{ex}{Exercice}
\newtheorem*{nota}{Notation}
\newtheorem*{rem}{Remarque}
\newtheorem*{rem2}{Remarques}
\newtheorem{de2}{Définition}
\newtheorem{pro2}[de2]{Propriété}
\newtheorem{thm2}[de2]{Théorème}

\setlength{\parindent}{0cm}
\setlength{\parskip}{1ex plus 0.5ex minus 0.2ex}
\newcommand{\hsp}{\hspace{20pt}}
\newcommand{\HRule}{\rule{\linewidth}{0.5mm}}

\usepackage{comment}
\usepackage{xr-hyper}


\title{}

\date{}
\begin{document}


\pagestyle{fancy}

\fancyhead{}
 \fancyfoot{}

 \lhead{ 2023/2024 \\  L3 Mathématiques
}
\chead{\textbf{ Calcul formel}\\} 
 \rhead{ Université de Lorraine \\  }

\newcommand{\lb}{\llbracket}
\newcommand{\rb}{\rrbracket}
\newcommand{\N}{\mathbb{N}}
\newcommand{\Z}{\mathbb{Z}}




\newcommand{\md}[3]{#1\ \equiv \ #2 \! \! \! \! \! \pmod {#3} }
\newcommand{\nmd}[3]{#1 \not \equiv #2 \! \! \! \! \!  \pmod {#3} }
\newcommand{\mda}[3]{#1 \equiv #2 \! \!  \pmod {#3} }
\newcommand{\nmda}[3]{#1 \not \equiv #2 \! \! \pmod {#3} }
\newcommand{\mo}[2]{#1 \! \! \! \! \! \pmod #2 }
\newcommand{\moa}[2]{#1 \! \!  \pmod {#2} }

\thispagestyle{fancy}

\begin{center}
%    \HRule \\[0.6cm]
    { \huge \bfseries
    DM 1
     \\ [0cm] }
    \HRule \\[0.5cm]
\end{center}






\begin{center}
\textbf{A rendre le 17 octobre}
\end{center}

Le but de ce devoir maison est d'étudier les idempotents de $\Z/n\Z$ (c'est à dire les $x\in \Z/n\Z$ tels que $x^2=x$), pour $n\in \N^*$. On commence par traiter quelques exemples, puis on traite le cas général. 


\begin{enumerate}
\item Déterminer les idempotents de $\Z/6\Z$ et $\Z/7\Z$. 


Soit $\phi:\Z/100\Z\rightarrow \Z/4\Z\times \Z/25\Z$ l'isomorphisme chinois (c'est à dire que $\phi(a\mathrm{\ mod\ }100)=(a\mathrm{\ mod\ }4, a\mathrm{\ mod\ }25)$, pour tout $a\in \Z$).


\item\label{inverse_theorem_chinois} En utilisant les deux algorithmes vus en cours, déterminer $\phi^{-1}\big((1\mathrm{\ mod\ }4,0\mathrm{\ mod\ }25)\big)$ et 

 $\phi^{-1}\big((0\mathrm{\ mod\ }4,1\mathrm{\ mod\ }25)\big)$.

\item\label{idempotent_explicite} Montrer que $\phi^{-1}\big((1\mathrm{\ mod\ }4,0\mathrm{\ mod\ }25)\big)$ et  $\phi^{-1}\big((0\mathrm{\ mod\ }4,1\mathrm{\ mod\ }25)\big)$ sont idempotents (on essaiera de limiter les calculs, qui peuvent se faire de tête en s'y prenant bien).




\item\label{inversibles} Décrire (faire la liste des éléments de) $(\Z/10\Z)^\times$ et $(\Z/16\Z)^\times$.

\item\label{idempotent_anneau} Soit $A$ un anneau commutatif unitaire et $x\in A$ un idempotent de $A$. Que peut-on dire de $x$ si  $x$ est inversible.

\item\label{idempotents10_16} Faire la liste des éléments idempotents de $\Z/10\Z$ et de $\Z/16\Z$.

\item\label{pgcd} Soit $x\in \Z$. Montrer que $x\wedge (x-1)=1$.


\item\label{caracterisation_idempotent_primaire} Soit $p\in \mathbb{P}$ et $\alpha\in \N^*$. Montrer que si $x\in \Z/p^\alpha\Z$ est idempotent, alors $x\in \{0\mathrm{\ mod\ }p^\alpha, 1\mathrm{\ mod\ }p^\alpha\}$ (on pourra utiliser la question précédente).




\bigskip

 Soit $n\in \N_{\geq 2}$. On suppose que la décomposition de $n$ en produits de facteurs premiers s'écrit $n=p_1^{\alpha_1}\ldots p_k^{\alpha_k}$, avec  $p_1,\ldots,p_k\in \mathbb{P}$, $p_1<\ldots < p_k$ et $\alpha_1,\ldots,\alpha_k\in \N^*$. Soit $f:\Z/n\Z\rightarrow \Z/p_1^{\alpha_1}\Z\times \ldots \times \Z/p_k^{\alpha_k}\Z$ l'isomorphisme chinois (c'est à dire que si $x \ \mathrm{mod}\ n\in \Z/n\Z$, $f(x\ \mathrm{mod}\ n)=(x\ \mathrm{mod}\ p_1^{\alpha_1},\ldots,x\ \mathrm{mod}\  p_k^{\alpha_k})$). 
 
\item\label{cardinal_idempotents} Soit $x\in \Z/n\Z$. Montrer que $x$ est idempotent si et seulement si $x\mathrm{\ mod\ }p_i^{\mathrm{\alpha_i}}\in \{0\mathrm{\ mod\ }p_i^{\alpha_i},1\mathrm{\ mod\ }p_i^{\alpha_i}\}$ pour tout $i\in\llbracket 1,k\rrbracket$ (on pourra considérer $f$ et utiliser la question~\ref{caracterisation_idempotent_primaire}). Combien $\Z/n\Z$ possède-t-il d'idempotents ?


\item\label{idempotents_100} Déterminer les idempotents de $\Z/100\Z$.



\end{enumerate}




\end{document}
