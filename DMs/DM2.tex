
\documentclass[11pt,a4paper]{article}
\usepackage[utf8]{inputenc}
\usepackage[T1]{fontenc}
\usepackage[french]{babel}
\usepackage[top=3cm, bottom=2cm, left=2cm, right=2cm]{geometry}
\usepackage{stmaryrd}
\usepackage{amsmath}
\usepackage{amsfonts}
\usepackage{amssymb}
\usepackage{mathrsfs}
\usepackage{amsthm}
\usepackage{layout}
\usepackage{fancyhdr}

\newtheorem*{thm}{Théorème}
\newtheorem{ex}{Exercice}
\newtheorem*{nota}{Notation}
\newtheorem*{rem}{Remarque}
\newtheorem*{rem2}{Remarques}
\newtheorem{de2}{Définition}
\newtheorem{pro2}[de2]{Propriété}
\newtheorem{thm2}[de2]{Théorème}

\setlength{\parindent}{0cm}
\setlength{\parskip}{1ex plus 0.5ex minus 0.2ex}
\newcommand{\hsp}{\hspace{20pt}}
\newcommand{\HRule}{\rule{\linewidth}{0.5mm}}

\title{}

\date{}
\begin{document}


\pagestyle{fancy}

\fancyhead{}
 \fancyfoot{}

 \lhead{ 2023/2024 \\  L3 Mathématiques
}
\chead{\textbf{ Calcul formel}\\} 
 \rhead{ Université de Lorraine \\  }

\newcommand{\lb}{\llbracket}
\newcommand{\rb}{\rrbracket}


\newcommand{\N}{\mathbb{N}}
\newcommand{\Z}{\mathbb{Z}}
\newcommand{\md}[3]{#1\ \equiv \ #2 \! \! \! \! \! \pmod {#3} }
\newcommand{\nmd}[3]{#1 \not \equiv #2 \! \! \! \! \!  \pmod {#3} }
\newcommand{\mda}[3]{#1 \equiv #2 \! \!  \pmod {#3} }
\newcommand{\nmda}[3]{#1 \not \equiv #2 \! \! \pmod {#3} }
\newcommand{\mo}[2]{#1 \! \! \! \! \! \pmod #2 }
\newcommand{\moa}[2]{#1 \! \!  \pmod {#2} }

\thispagestyle{fancy}

\begin{center}
%    \HRule \\[0.6cm]
    { \huge \bfseries
    DM 
     \\ [0cm] }
    \HRule \\[0.5cm]
\end{center}





\begin{center}
\textbf{A rendre  le mardi 28 novembre}
\end{center}

\

L'objet de ce devoir est de démontrer le théorème de Korselt, qui caractérise les nombres de Carmichael. Rappelons qu'un nombre $n\in \N_{\geq 2}$ est de Carmichael si pour tout $a\in \Z$ tel que $a\wedge n=1$, on a $a^{n-1}\equiv 1[n]$. 

\textbf{Théorème (Korselt)} : Soit $n\in \N_{\geq 2}$ composé. Alors $n$ est de Carmichael si et seulement si les conditions suivantes sont satisfaites : 
\begin{enumerate}
\item $n$ est sans facteurs carrés (pour tout $k\in \N_{\geq 2}$, $k^2$ ne divise pas $n$)
\item pour tout diviseur premier $p$ de $n$, $p-1$ divise $n-1$. 
\end{enumerate}




\begin{itemize}
\item[$1.a)$] Soit $x\in  \Z$, $k\in \N^*$, $p_1,\ldots,p_k$ des nombres premiers distincts, $\alpha_1,\ldots,\alpha_k\in \N^*$ et $n=p_1^{\alpha_1}\ldots p_k^{\alpha_k}$. Montrer que $x\equiv 1[n]$ si et seulement si ($x\equiv 1[p_i^{\alpha_i}]$, pour tout $i\in \llbracket 1,k\rrbracket$).

\item[$1.b)$] Montrer que si $n = p_1  p_2 \ldots  p_k$ où $k\geq 2$ et $p_1,\ p_2,\ \ldots, p_k$ sont des nombres premiers deux à deux
distincts tels que $(p_i - 1)$ divise $(n - 1)$ pour tout $i\in \lb 1; k\rb$, alors $n$ est un nombre de Carmichaël.
Montrer en particulier que $561$ et $10585$ sont des nombres de Carmichaël.\\
\item[$2.$] Dans toute cette question, on suppose que $n$ est un nombre de Carmichaël et l'on désire établir la réciproque du résultat obtenu en question $1$ b).\\
\begin{itemize}
\item[$a)$] On suppose tout d'abord que $n$ est une puissance de $2$, $n = 2^{\alpha}$, où $\alpha$ est un entier supérieur à $2$. Quel est le cardinal de $(\mathbb{Z}/n\mathbb{Z})^{\times}$ ? En déduire que pour tout entier $a$ impair $a^{2^{\alpha} - 1}$ ne peut
être congru à $1$ modulo $n$ sauf si $a$ est congru à $1$ modulo $n$; que peut-on conclure ?\\
\item[$b)$] On suppose désormais que $n$ admet au moins un facteur premier impair $p_1$ et l'on note $p_1, p_2,\ldots, p_k$ les facteurs premiers de $n$ ; la décomposition de $n$ est alors $n =\prod_{i=1}^k p_i^{\alpha_i}$.\\
\begin{itemize}
\item[$(i)$] Soit $\omega$ un entier dont la classe modulo $p_1^{\alpha_1}$ est un générateur de $((\mathbb{Z}/p_1^{\alpha_1}
\mathbb{Z})^{\times}, \times)$ ; on admet l'existence d'un tel $\omega$. En utilisant le théorème des restes chinois, montrer qu'on peut trouver un entier $t$ tel que :\\
$t \equiv \omega [p_1^{\alpha_1}]$ et, pour tout $i$ (s'il en existe) tel que $i \in \lb 2;k\rb$, $t \equiv 1 [p_i^{\alpha_i}]$.\\
Montrer qu'alors $t^{n-1}\equiv 1 [n]$. \\
\item[$(ii)$] En déduire que $p_1^{\alpha_1-1}(p_1 - 1)$ divise $(n- 1)$, puis que $\alpha_1 = 1$, et enfin que $(p_1 - 1)$ divise $(n - 1)$. En déduire que $n$ est impair.\\
\item[$(iii)$] Montrer que $n$  peut s'écrire sous la forme $n = p_1 \times p_2 \times \ldots \times p_k$ où $p_1, p_2, \ldots, p_k$ sont des nombres premiers deux à deux distincts tels que $(p_i - 1)$ divise $(n - 1)$ pour tout $i \in \lb 1,k \rb $. Conclure.\\
\end{itemize}
\end{itemize}
\item[$3.$] Montrer qu'un nombre de Carmichaël admet au moins trois facteurs premiers.\\
\item[$4.$] Résoudre l'équation $85p - 16k = 1$, où $(k, p)\in \mathbb{Z}^2$.\\
Déterminer le plus petit nombre de Carmichaël divisible par $5$ et $17$.\\

\item On veut montrer le critère de Solovay-Strassen: soit $n\in \N_{\geq 3}$ impair. Alors $n$ est premier si et seulement si \begin{equation}\label{e_critere}
a^{n-1}\equiv a^{(n-1)/2} [n],\forall a\in \Z\mid a\wedge n =1.
\end{equation} On a vu en cours le sens $\Rightarrow$. On veut donc montrer la réciproque. On fixe donc $n$ satisfaisant \eqref{e_critere}. On suppose que $n$ n'est pas premier.

\begin{enumerate}
\item[ a)] Montrer que $n$ est de Carmichael.

\item[ b)] En utilisant le théorème de Korselt, on écrit $n=p_1\ldots p_k$, où $k\in \N$ et  $p_1,\ldots,p_k\in \mathbb{P}_{\geq 3}$.  Soit $a\in \Z$ tel que $a\wedge n=1$. Si $i\in \llbracket 1,k\rrbracket$, on note $r_i=r_i(a)$ le reste dans la division euclidienne de $a$ par $p_i$. Si $x\in \Z$ et $\ell\in \Z$, on note $x[\ell]$ son image dans $\Z/\ell\Z$.  Montrer que \begin{equation}\label{e_rel}
a^{(n-1)/2}[p_1]=\left(\frac{r_1}{p_1}\right)\ldots \left(\frac{r_k}{p_k}\right)[p_1].
\end{equation}

\item[c)] Aboutir à une contradiction (on pourra utiliser le théorème chinois).



Si $a\in \Z$ est premier avec $n$, on a $\left(\frac{a}{n}\right)\in \{-1,1\}$, donc $\left(\frac{a}{n}\right)^2=1$. Soit $a\in \Z$ tel que $a\wedge n=1$. Alors $a^{n-1}=(a^{(n-1)/2})^2\equiv \left(\frac{a}{n}\right)^2=1[n]$, par hypothèse. On en déduit que $n$ est de Carmichael.

 On a  $\left(\frac{a}{n}\right)=\left(\frac{a}{p_1}\right)\ldots \left(\frac{a}{p_k}\right)$.  Soit $i\in \llbracket 1,k\rrbracket$. Alors $\left(\frac{a}{p_i}\right)=\left(\frac{r_i}{p_i}\right)$, d'où \eqref{e_rel}. 
 
 Le terme de gauche de \eqref{e_rel} dépend uniquement de $a[p_1]$, alors que le terme de droite dépend de $a[p_2],\ldots, a[p_k]$. Soit $f:\Z/n\Z\rightarrow \Z/p_1\Z\times \ldots \Z/p_r\Z$ l'isomorphisme chinois. Soient $x\in \Z$ tel que $x$ n'est pas un carré modulo $p_2$ (on sait qu'un tel élément existe car $p_2\neq 2$ et qu'il existe exactement $(p_2+1)/2$ carrés modulo $p_2$). Soient $b,c\in \Z$ tels que   $b[n]=f^{-1}(r_1[p_1],x[p_2],1[p_3],\ldots,1[p_k])$ et $c[n]=f^{-1}(r_1[p_1],x[p_2],1[p_3],\ldots,1[p_k])$. Alors $a^{(n-1)/2}[p_1]=b^{(n-1)/2}[p_1]=c^{(n-1)/2}[p_1]$, $r_i(a)=r_i(b)=r_i(c)$ pour tout $i\in \llbracket 1,k\rrbracket\setminus \{2\}$. En revanche $-1=\left(\frac{r_2(b)}{p_2}\right)\neq \left(\frac{r_2(c)}{p_2}\right)=1$. On a donc $\left(\frac{b}{n}\right)\neq \left(\frac{c}{n}\right)$, donc \eqref{e_rel} n'est pas satisfaite pour $b$ et $c$ simultanément. On aboutit à une contradiction.















\end{enumerate}


%\item[$5.$] Montrer que si les nombres $6k+1$, $12k+1$ et $18k+1$ sont tous les trois premiers, alors leur produit est un nombre de Carmichael.
\end{itemize}

















\end{document}
